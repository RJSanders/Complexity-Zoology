\documentclass[12pt]{article}

\usepackage[margin=1in]{geometry}
\usepackage{lmodern}
\usepackage{amsmath,amsthm,amssymb,mathrsfs}
\usepackage{enumerate}
\usepackage{listings}
\usepackage{url}

\title{Complexity Zoology}
\author{Robert Sanders}

\newtheorem{theorem}{Theorem}[section]
\newtheorem{lemma}[theorem]{Lemma}
\newtheorem{proposition}[theorem]{Proposition}
\newtheorem{corollary}[theorem]{Corollary}

\theoremstyle{definition}
\newtheorem{definition}[theorem]{Definition}

\theoremstyle{remark}
\newtheorem{claim}[theorem]{Claim}
\newtheorem{remark}[theorem]{Remark}

\newcommand{\C}{\mathbb{C}}
\newcommand{\N}{\mathbb{N}}
\newcommand{\Q}{\mathbb{Q}}
\newcommand{\R}{\mathbb{R}}
\newcommand{\Z}{\mathbb{Z}}
\newcommand{\ra}{\rightarrow}
\newcommand{\cL}{\mathcal{L}}
\newcommand{\start}{\mathtt{start}}
\newcommand{\halt}{\mathtt{halt}}
\newcommand{\Left}{\mathsf{L}}
\newcommand{\Stay}{\mathsf{S}}
\newcommand{\Right}{\mathsf{R}}
\newcommand{\A}{\mathsf{A}}
\newcommand{\co}{\mathsf{co}}
\newcommand{\cocap}{\mathsf{cocap}}
\newcommand{\NP}{\mathsf{NP}}
\newcommand{\sP}{\mathsf{P}}
\newcommand{\poly}{\mathsf{poly}}
\newcommand{\msf}[1]{\mathsf{#1}}
\newcommand{\id}{\msf{id}}
\newcommand{\AND}{\mathbin{\&}}
\newcommand{\Oh}{\mathcal{O}}
\newcommand{\inner}[1]{\left\langle#1\right\rangle}

\begin{document}
\maketitle

\section{Overview and Preliminaries}

Complexity Zoology is an expert system for studying inclusions and oracle
separations of complexity classes. The system reads a text file describing an
initial set of inclusions and separations, then deduces all logical
consequences. For a potential inclusion $\msf{A}\subseteq\msf{B}$, the system is
capable of understanding that the inclusion is true, false, or unknown relative
to (a) all oracles, (b) the random oracle, (c) the trivial oracle, (d) all
affine oracles (see \cite{aydinlioglu2018affine}), (e) some affine oracle, or
(f) an arbitrary oracle.

We use the notation $\Sigma=\{0,1\}$, and for a set $S$ we use $S^*$ to indicate
the set of all finite sequences in $S$. For definiteness, meanings of standard
concepts in complexity theory, such as Turing machines and commonly arising
complexity classes are taken from \cite{arora2009computational}.

\section{Complexity Classes and Oracles}

For the purposes of this system, a complexity class is not a set of languages
but a family of sets of languages indexed by the set of all oracles, where
oracles are considered to be decision functions $f:\Sigma^*\ra\Sigma$. For
example, we consider $\sP$ to be the class
$\sP=\{\sP^f:f:\Sigma^*\ra\Sigma\}$,
where $\sP^f$ is $\sP$ relative to the oracle $f$. This allows for a
consistent notion of what it means for an inclusion or separation to hold for
all, none, or some oracles, avoiding the need for a universal definition of
$\A^f$ for an arbitrary complexity class $\A$ and an oracle $f$.

If a complexity class $\A$ is defined in terms of Turing machines, then $\A^f$
can be defined in the same way, replacing references to Turing machines with
references to oracle Turing machines that query the $f$-oracle. Formally, let
$TM$ denote the set of all Turing machines, and let $TM_f$ denote the set of all
$f$-oracle Turing machines. Then if $\A$ is defined to be
\[
\A=\{L\subseteq\Sigma^*:(\exists M\in TM)\varphi(L,M)\}
\]
for a binary predicate $\varphi(x,y)$, we consider $\A^f$ to be defined to be
\[
\A^f=\{L\subseteq\Sigma^*:(\exists M\in TM_f)\varphi(L,M)\}.
\]

\section{Operators on Complexity Classes}

An \textit{operator} on the set of all complexity classes is an
inclusion-preserving automorphism.

We first define a complexity class operator that, while not implemented in
Complexity Zoology, is necessary for defining a key property of the classes in
the system.
\begin{definition}
For a complexity class $\A$, define
\[
\msf{P_{prep}}\cdot\A=\{f(\cL):\cL\in\A\AND f:\Sigma^*\ra\Sigma^*\text{ is
computable in polynomial-time.}\}.
\]
\end{definition}
For the complexity classes in this project, it is the case that
$\msf{P_{prep}}\cdot\A=\A$. The operator inclusions and quadratic relations
that the system uses for knowledge propogation can fail if this condition is not
satisfied.

\begin{definition}
The operators $\id$, $\co$, and $\cocap$ are given by
\begin{align*}
\id\cdot\A&=\A, \\
\co\cdot\A&=\{\cL\subseteq\Sigma^*:\Sigma^*\setminus\cL\in\A\}, \\
\cocap\cdot\A&=\A\cap\co\cdot\A.
\end{align*}
\end{definition}

It is immediate from these definitions that $\co\cdot\co\cdot\A=\A$,
$\cocap\cdot\A\subseteq\A$, and $\cocap\cdot\A\subseteq\co\cdot\A$ for every
$\A$. Additionally, the $\cocap$ operator absorbs the $\co$ operator, because
\begin{align*}
\cocap\cdot\co\cdot\A&=(\co\cdot\A)\cap(\co\cdot\co\cdot\A) \\
&=\A\cap\co\cdot\A \\
&=\cocap\cdot\A
\end{align*}
and
\begin{align*}
\co\cdot\cocap\cdot\A&=\{\cL\subseteq\Sigma^*:\Sigma^*\setminus\cL\in\cocap\cdot
\A\} \\
&=\{\cL\subseteq\Sigma^*:\Sigma^*\setminus\cL\in\A
\AND\Sigma^*\setminus\cL\in\co\cdot\A\} \\
&=\{\cL\subseteq\Sigma^*:\cL\in\co\cdot\A\AND\cL\in\A\} \\
&=\cocap\cdot\A.
\end{align*}

It follows that $\cocap$ is idempotent, because
$\cocap\cdot\cocap\cdot\A=\cocap\cdot\A\cap\co\cdot\cocap\cdot\A=\cocap\cdot\A$.

The next operator defines \textit{polynomial advice}.

\begin{definition}
For a complexity class $\A$, define
\[
\poly(\A)=\{\cL\subseteq\Sigma^*:(\exists\cL^\prime\in\A, f:\N\ra\Sigma^*)
[|f(n)|=\Oh(n^k)\AND(x\in\cL\Leftrightarrow\langle
x,f(|x|)\rangle\in\cL^\prime)]\}.
\]
\end{definition}
For the models of computation in this project, it is possible to ignore the
advice string $f(n)$, so in practice $\A\subseteq\poly\cdot\A$ for each
complexity class $\A$ of interest.

Furthermore, observe that $\cL\subseteq\poly\cdot\poly\cdot\A$ if and only if
there exist $\cL^\prime\in\A$ and $f,g:\N\ra\Sigma^*$ with
$|f(n)|,|g(n)|=\Oh(n^k)$ such that
\[
x\in\cL\Longleftrightarrow
\inner{\inner{x,f(|x|)},g(|\inner{x,f(|x|)}|)}\in\cL^\prime.
\]
The models of computation we consider are unaffected by slight changes to the
encoding of tuples, so for the complexity classes studied here, the above is
equivalent to the existence of $\cL^\prime,f,g$ such that
\[
x\in\cL\Longleftrightarrow
\langle x,
\underbrace{\inner{f(|x|),g(|\inner{x,f(|x|)}|)}}_{h(|x|)}\rangle\in\cL^\prime.
\]
Then $h:\N\ra\Sigma^*$ satisfies $|h(n)|=\Oh(n^k)$, and hence
$\cL\in\poly\cdot\poly\cdot\A\Rightarrow\cL\in\poly\cdot\A$. Thus, we also
regard $\poly$ as an idempotent operator.

The remaining operators are well defined for all complexity classes but are
intended for those involving polynomial-time computations.

\begin{definition}
For a complexity class $\A$, say that $\cL\in\msf{N}\cdot\A$ if there is exists a
polynomial $p:\N\ra(0,\infty)$ and a language $\cL^\prime\in\A$ such that
\begin{align*}
x\in\cL&\Longleftrightarrow\inner{x,y}\in\cL^\prime\text{ for some }y
\in\Sigma^{p(|x)}.
\end{align*}
\end{definition}

\begin{definition}
For a complexity class $\A$, a language $\cL$ lies in $\oplus\cdot\A$ if there
exists a language $\cL^\prime\in\A$ and a function $p:\N\ra(0,\infty)$
satisfying $p(n)=\Oh(n^k)$ for some positive integer $k$ such that for every
$x\in\Sigma^*$,
\[
x\in\cL\Longleftrightarrow|\{y\in\Sigma^{p(|x|)}:\inner{x,y}\in\cL^\prime\}|
\equiv 1\text{ (mod 2)}.
\]
\end{definition}

$\msf{BP}$ and $\msf{P}$ are the probabilistic operators:

\begin{definition}
For a complexity class $\A$, say that $\cL\in\msf{BP}\cdot\A$ if there is
exists a polynomial $p:\N\ra(0,\infty)$ and a language $\cL^\prime\in\A$ such
that
\begin{align*}
x\in\cL&\Longrightarrow\inner{x,y}\in\cL^\prime\text{ for }>2/3\text{ of all
 }y\in\Sigma^{p(|x)}; \\
x\notin\cL&\Longrightarrow\inner{x,y}\notin\cL^\prime\text{ for }>2/3\text{
of all }y\in\Sigma^{p(|x)}.
\end{align*}
The $\msf{P}$ is defined similarly, except that $2/3$ is replaced with $1/2$.
\end{definition}

\section{Annotations}

This list consists of annotations used in the input files of Complexity
Zoology. These annotations denote recurring footnotes too small for their own
section in this document.

\begin{itemize}

\item \textbf{[count]}: The complexity classes can be separated because they
have different cardinalities.
\item \textbf{[def]}: This statement is true by definition.
\item \textbf{[imm]}: This result is immediate from the definitions of the
respective complexity classes.
\item \textbf{[probably]}: This result is believed to be true, but it needs to
be checked.
\item \textbf{[rel?]}: It should be double-checked that this statement
relativizes.

\end{itemize}

\section{Diagonalization (diag)}

The method of diagonalization can be used to unconditionally separate several
complexity classes. In its standard form, it can be used to prove the
\textit{time hierarchy theorem}.

\begin{theorem}
If $f,g:\N\ra\N$ are time-constructible functions satisfying
$f(n)\log f(n)=o(g(n))$, then
\[
\msf{DTIME}(f(n))\subsetneq\msf{DTIME}(g(n)).
\]
\end{theorem}
To prove this theorem, construct a TM $D$ that carries out the following
procedure: on input $x$, compute $M_x(x)$ on a suitable universal TM for
$g(|x|)$ steps, where $M_x$ is the Turing machine encoded by $x$. If the
computation finishes, output $1-M_x(x)$. Otherwise, output $0$. Next, assume the
language $\cL$ that $D$ determines lies in $\msf{DTIME}(f(n))$. Then, there
exists a TM $M$ that decides $\cL$ in $\Oh(f(n))$-time. Then $M$ has some
encoding, and in fact can be assumed to have infinitely many encodings, so we
fix some encoding $y$ that is long enough so that $f(|y|)\log f(|y|)$ is much
less than $g(|y|)$. Then the universal Turing machine can simulate $M_y$ on
input $y$ within $g(|y|)$ steps, and so $D(y)=1-M_y(y)=1-M(y)$. But since $D$
and $M$ both decide $\cL$, we should have $D(y)=M(y)$, so this is a
contradiction.

\section{$\msf{AWPP}\subseteq\msf{PP}$(AWPPvPP)}

$\msf{AWPP}$ can be defined in terms of functions in $\msf{GapP}$; i.e.,
functions $f:\Sigma^*\ra\Z$ such that for a non-deterministic TM $M$, $f(x)$ is
the difference between the number of accepting paths and the number of rejecting
paths for $M$ with input $x$ \cite{fenner1994gap}. Specifically, a language
$\cL$ lies in $\msf{AWPP}$ if and only if for every polynomial $p$, there exists
a function $f\in\msf{GapP}$ and an everywhere nonzero function $g:\Sigma^*\ra\N$
in $\msf{FP}$ such that
\begin{align*}
x\in\cL&\Longrightarrow 1-2^{-p(|x|)}\leq f(x)/g(x)\leq 1, \\
x\notin\cL&\Longrightarrow 0\leq f(x)/g(x)\leq 2^{-p(|x|)}.
\end{align*}
Also, note that $\mathcal{L}\in\msf{PP}$ if and only if there is a function
$f\in\msf{GapP}$ such that
\begin{align*}
x\in\cL&\Longrightarrow f(x)\geq 0, \\
x\notin\cL&\Longrightarrow f(x)<0.
\end{align*}
Suppose that $\cL\in\msf{GapP}$. Then fix $p=2$ in the definition of
$\msf{AWPP}$; there must exist functions $f\in\msf{GapP}$ and $g:\Sigma^*\ra\N$
in $\msf{FP}$ such that
\begin{align*}
x\in\cL&\Longrightarrow 3/4\leq f(x)/g(x)\leq 1, \\
x\notin\cL&\Longrightarrow 0\leq f(x)/g(x)\leq 1/4.
\end{align*}
Since $\msf{FP}\subseteq\msf{GapP}$ and $\msf{GapP}$ is closed under
multiplication and subtraction, $h(x)=4f(x)-2g(x)$ is in $\msf{GapP}$, and hence
\begin{align*}
x\in\cL&\Longrightarrow g(x)\leq 4f(x)-2g(x)\leq 2g(x)
\Longrightarrow h(x)\geq 0, \\
x\notin\cL&\Longrightarrow -2g(x)\leq 4f(x)-2g(x)\leq-g(x)
\Longrightarrow h(x)< 0.
\end{align*}
So $\cL\in\msf{PP}$, and we have $\msf{AWPP}\subseteq\msf{PP}$.

\section{Oracle Access (oap)}

For a pair of classes $\A$ and $\msf{B}$, it may be the case that
$\A^O\not\subseteq\msf{B}^O$ because the computational model of $\A^O$ is able
to make longer oracle calls than that of $\msf{B}^O$. For example, $\msf{B}^O$
could be polynomially limited in space or time and therefore be unable to write
long questions for the oracle, while $A^O$ has no such limitations.

\section{Password Oracles (pass)}

A \textit{password oracle} is a type of oracle $f$ constructed so that
$\A^f\not\subseteq\msf{B}^f$. Typically, $f$ is a function
$\Sigma^*\times\Sigma^*\rightarrow\Sigma^*$ chosen so that
$PW_f=\{x\in\Sigma:P\}$ lies in $\A^f$ but not in $\msf{B}^f$, where $P$ is a
proposition depending on the values of $f(x,y)$ for $y\in\Sigma^*$ (the
\textit{passwords} of $x$). The oracle $f$ can be adversarially constructed or,
in many cases, selected according to a random process.

\begin{theorem}
Let $f:\Sigma^{2*}\ra\Sigma\cup\{\square\}$ be a function selected randomly
according to the following rules:
\begin{itemize}
\item For every $x\in\Sigma^n$, there exists a unique $y\in\Sigma^n$ such that
$f(x,y)\neq\square$. This $y$ is selected using the uniform distribution on
$\Sigma^n$.
\item  For every $x\in\Sigma^n$, if $y$ is the unique element of $\Sigma^n$ such
that $f(x,y)\neq\square$, then $\Pr[f(x,y)=1]=\Pr[f(x,y)=0]=1/2$.
\end{itemize}
Then $(\cocap\cdot\msf{UP})^f\not\subseteq(\sP/\poly)^f$ with probability 1.
\end{theorem}

\begin{proof}
For each $x\in\Sigma^*$, define $PW_f(x)=f(x,y)$, where $y$ is the unique
element of $\Sigma^{|x|}$ such that $f(x,y)\neq\square$. Then
$PW_f\in(\cocap\cdot\msf{UP})^f$, because for a given $x\in\Sigma^*$ the unique
$y$ can be used as a the certificate to check that $PW_f(x)=1$ or $PW_f(x)=0$.

Fix an enumeration $\{M_k\}$ of Turing machines. For $M_k$ and input of length
$n$, we allow computation times up to $C_kn^{r_k}$ and advice strings up to
length $D_kn^{s_k}$, where the coefficients and exponents are unbounded and
increasing as functions of $k$. Then, since for any $x\in\Sigma^n$ there are
$2^n$ possible values of $y$, while advice and computation time are polynomials
of $n$, we have
\[
\Pr[(\forall n)(\exists\text{ advice }a)(\forall|x|=n)[M_k(x,a)=PW_f(x)]]=0.
\]
\end{proof}

\section{$\msf{ZPP} = \cocap\cdot\msf{RP}$ (ZRP)}

The equality $\msf{ZPP} = \cocap\cdot\msf{RP}$ is sometimes taken to be the
definition of $\msf{ZPP}$. Otherwise, $\msf{ZPP}$ is defined as the class of
decision problems computable by a probabilistic Turing machine that always
computes the correct solution and is expected to run in polynomial-time. With
this definition, one can show $\msf{ZPP}\subseteq\msf{RP}$ by running a
$\msf{ZPP}$-machine for twice its expected running time, then returning the
output 0 if an answer has not yet been determined. Since $\msf{ZPP}$ is
symmetric, it follows that $\msf{ZPP}\subseteq\cocap\cdot\msf{RP}$. Conversely, to
show that $\cocap\cdot\msf{RP}\subseteq\msf{ZPP}$, run an $\msf{RP}$- and
$\co\cdot\msf{RP}$-machine in parallel, and repeat as necessary until either the
$\msf{RP}$-machine returns 1 or the $\co\cdot\msf{RP}$-machine returns 0.

\section{Bibliography}

\nocite{*}
\bibliographystyle{alpha}
\bibliography{refs}


\end{document}
