\documentclass[12pt]{amsart}

\usepackage[margin=1in]{geometry}
\usepackage{mathptmx}
\usepackage{amsmath,amsthm,amssymb,mathrsfs}
\usepackage{enumerate}
\usepackage{listings}
\usepackage{url}

\title{Complexity Class Operators}
\date{}
\author{Robert Sanders}

\newtheorem*{theorem}{Theorem}
\newtheorem*{lemma}{Lemma}
\newtheorem*{proposition}{Proposition}
\newtheorem*{corollary}{Corollary}

\theoremstyle{definition}
\newtheorem*{definition}{Definition}

\theoremstyle{remark}
\newtheorem*{claim}{Claim}
\newtheorem*{remark}{Remark}

\newcommand{\C}{\mathbb{C}}
\newcommand{\N}{\mathbb{N}}
\newcommand{\Q}{\mathbb{Q}}
\newcommand{\R}{\mathbb{R}}
\newcommand{\Z}{\mathbb{Z}}
\newcommand{\ra}{\rightarrow}
\newcommand{\cL}{\mathcal{L}}
\newcommand{\start}{\mathtt{start}}
\newcommand{\halt}{\mathtt{halt}}
\newcommand{\Left}{\mathsf{L}}
\newcommand{\Stay}{\mathsf{S}}
\newcommand{\Right}{\mathsf{R}}
\newcommand{\A}{\mathsf{A}}
\newcommand{\sC}{\mathsf{C}}
\newcommand{\co}{\mathsf{co}}
\newcommand{\cocap}{\mathsf{cocap}}
\newcommand{\NP}{\mathsf{NP}}
\newcommand{\sP}{\mathsf{P}}
\newcommand{\poly}{\mathsf{poly}}
\newcommand{\msf}[1]{\mathsf{#1}}
\newcommand{\id}{\msf{id}}
\newcommand{\AND}{\mathbin{\&}}
\newcommand{\OR}{\mathbin{\text{or}}}
\newcommand{\Oh}{\mathcal{O}}
\newcommand{\inner}[1]{\left\langle#1\right\rangle}
\newcommand{\op}{\msf{op}}
\newcommand{\sN}{\msf{N}}
\newcommand{\BP}{\msf{BP}}
\newcommand{\exppad}{\msf{exppad}}
\newcommand{\FP}{\msf{FP}}

\begin{document}
\maketitle

A \textit{complexity class operator} $\op$ is an inclusion-preserving
automorphism on the set of all complexity classes, written as $\op\cdot\sC$ for
a complexity class $\sC$. Thus, if $\sC\subseteq\msf{D}$, then
$\op\cdot\sC\subseteq\op\cdot\msf{D}$. Complexity Zoology's knowledge of
operators consists of inequalities of the form $\op_1\leq\op_2$, meaning that
$(\op_1\cdot\sC)^A\subseteq(\op_2\cdot\sC)^A$ for each class $\sC$ and oracle
$A$, and quadratic relations of the form $\op_1\cdot\op_2=\op_3\cdot\op_4$,
meaning that $(\op_1\cdot\op_2\cdot\sC)^A=(\op_3\cdot\op_4\cdot\sC)^A$ for each
class $\sC$ and oracle $A$.

\section{Definitions}

The definitions of the following complexity
class operators preserve relativization. In other words, if an operator
$\msf{op}$ is defined by the property that
\[
  \msf{op}\cdot\sC=\{\cL\subseteq\Sigma^*:\varphi(\cL,\sC)\}
\]
for any complexity class $\sC$, then the relativized version of
$\msf{op}\cdot\sC$ is
\[
  (\msf{op}\cdot\sC)^A=\{\cL\subseteq\Sigma^*:\varphi(\cL,\sC^A)\}.
\]

The simplest operators are $\id$, the identity operator; $\co$, which swaps
``yes'' and ``no'' answers to each decision problem; and $\cocap$, which takes
the intersection of a class with its complement.

\begin{definition} For each complexity class $\sC$, we set
\begin{align*}
  \id\cdot\sC&:=\{\cL\subseteq\Sigma^*:\cL\in\sC\}=\sC, \\
  \co\cdot\sC&:=\{\cL\subseteq\Sigma^*:\Sigma^*\setminus\cL\in\sC\}, \\
  \cocap\cdot\sC&:=\{\cL\subseteq\Sigma^*:\cL\in\sC\AND\cL\in\co\cdot\sC\}=
                  \sC\cap(\co\cdot\sC).
\end{align*}
A class is \textbf{symmetric} if $\sC=\co\cdot\sC$ with respect to every oracle.
\end{definition}
For example, the class $\sP$ is symmetric, while $\NP$ is not, because there is
an oracle $A$ relative to which $\NP^A\neq\co\NP^A$ (although, of course, this
is an open problem in the absence of an oracle.

The $\poly$ operator adds a polynomial-length advice string to each input. To be
precise, we denote by $|\Oh(n^*)|$ the set of all functions $p:\N\ra\Sigma^*$
such that $|p(n)|=\Oh(n^k)$ for some $k\in\N$. A function $p\in|\Oh(n^*)|$ is
regarded as an \textit{advice function} when $p(n)$ is given as an argument for
each input of length $n$.
\begin{definition}
For a complexity class $\sC$, we define
\[
  \poly\cdot\sC=\{\cL\subseteq\Sigma^*:(\exists \cL^\prime\in\sC,
  p\in|\Oh(n^*)|)(\forall x\in\Sigma^*)[x\in\cL\Longleftrightarrow
  \inner{x,p(|x|)}\in\cL^\prime]\}.
\]
\end{definition}
We allow advice functions to map to the null string $\epsilon$ of length
zero. In the case of tuples, $\inner{x,\epsilon}$ should be understood to be
$x$, so that, as we will see, $\poly\cdot\sC$ always contains $\sC$. For most
classes with polynomial advice, we write $\poly\cdot\sC=\sC/\poly$; we have, for
instance, $\sP/\poly$, $\NP/\poly$, and $\msf{BQP}/\poly$.

The operators $\oplus$, $\sN$, and $\sP$ are all defined in terms of
certificates, strings whose lengths are polynomials of the length of the
original input. Let $\Oh(n^*)$ denote the set of all functions $p:\N\ra\N$ such
that $p(n)=\Oh(n^k)$ for some $k\in\N$; then, for a quantifier $Q$, we can
define an operator $\op_Q$ by
\[
  \op_Q\cdot\sC:=\{\cL\subseteq\Sigma^*:
  (\exists\cL^\prime\in\sC, p\in\Oh(n^*))(\forall x\in\Sigma^*)
  [x\in\cL\Longleftrightarrow(Qy\in\Sigma^{p(|x|)})[\inner{x,y}\in\cL^\prime]]\}.
\]
The aforementioned operators are then equal to $\op_Q$ for different choices of
$Q$.
\begin{definition}
The operators $\oplus$, $\sN$, and $\sP$ are defined as follows for a
complexity class $\sC$:
\begin{itemize}
\item $\oplus\cdot\sC:=\op_Q\cdot\sC$, where $(Qy\in S)$ means ``for an odd
  number of $y\in S$.''
\item $\sN\cdot\sC:=\op_Q\cdot\sC$, where $(Qy\in S)$ means $(\exists y\in S)$.
\item $\sP\cdot\sC:=\op_Q\cdot\sC$, where $(Qy\in S)$ means ``for more than 1/2
  of all $y\in S$.''
\end{itemize}
$\oplus$ is read as ``parity.''
\end{definition}

The bounded probabilistic operator $\BP$ is defined similarly.
\begin{definition}
For each complexity class $\sC$,
\begin{align*}
\BP\cdot\sC:=\{\cL\subseteq\Sigma^*:&(\exists\cL^\prime,p\in\Oh(n^*))(\forall
x\in\Sigma^*)[[x\in\cL\Longrightarrow
(\exists_{>2/3}\hspace{2pt}y\in\Sigma^{p(|x|)})[\inner{x,y}\in\cL^\prime]] \\
&\AND[x\notin\cL\Longrightarrow(\exists_{>2/3}\hspace{2pt}y\in\Sigma^{p(|x|)})
[\inner{x,y}\notin\cL^\prime]]]\},
\end{align*}
where $(\exists_{>2/3}\hspace{2pt}y\in\Sigma^{p(|x|)})$ is understood to
mean ``for more than 2/3 of all $y\in\Sigma^{p(|x|)}$.''
\end{definition}
All of these operators are named in such a way that they suggest the definitions
of common complexity classes; for example, $\NP=\sN\cdot\sP$,
$\sP\sP=\sP\cdot\sP$, $\msf{BPP}=\BP\cdot\sP$, and $\oplus\sP=\oplus\cdot\sP$.

Finally, we have the operator $\sC\mapsto\sP^\sC$, which maps to $\sC$ to $\sP$
with $\sC$-oracle, as well as $\exppad$, which adds an exponential length of
zeros to input, generally for the purpose of buying additional computational
time.
\begin{definition}
For every complexity class $\sC$,
\[
\sP^\sC:=\{\cL\subseteq\Sigma^*:(\exists A\in C)[\cL\in\sP^A]\}=\bigcup_{A\in
  C}\sP^A,
\]
where the languages in $\sC$ have been identified with their decision
functions.
\end{definition}
The above operator is used in the definition of the polynomial hierarchy: for
every $n\in\N$, $\Delta_{n+1}^\sP=\sP^{\Sigma_n^\sP}$.
\begin{definition}
Let $\Oh(2^\text{poly})$ denote the set of all functions $f:\N\ra\N$ such that
$f(n)=\Oh(2^{p(n)})$ for some polynomial function $p:\N\ra\N$. Then, for a
complexity class $\sC$,
\[
\exppad\cdot\sC:=\{\cL\subseteq\Sigma^*:(\exists\cL^\prime\in\sC,f\in
\Oh(2^\text{poly}))[x\in\cL\Longleftrightarrow x0^{f(|x|)}\in\cL^\prime]\}.
\]
\end{definition}
$\msf{NEXP}$, for example, is not defined to  be $\sN\cdot\msf{EXP}$, but rather
$\exppad\cdot\NP$.

\section{Properties of Complexity Classes}

Proving the properties of complexity operators often requires that the
underlying complexity classes themselves have certain regularity
properties. First, every complexity class of interest should be
\textit{nontrivial} in the sense that it contains a nonempty language not equal
to $\Sigma^*$. We also expect that if $\cL\in\sC$, then any languages that are
reducible to $\cL$ in polynomial-time are also in $\sC$.

\begin{definition}
A complexity class $\sC$ is \textbf{polynomial-time self-reducible} if for every
$\cL\in\sC$ and every function $f\in\FP$,
\[
f^{-1}[\cL]=\{x\in\Sigma^*:f(x)\in\cL\}\in\sC.
\]
\end{definition}

Every class in this project is relativizingly nontrivial and polynomial-time
self-reducible, so that for each oracle $A$, if $f\in\FP^A$ and $\cL\in\sC^A$
then $f^{-1}[\cL]\in\sC^A$. As a result, $\sP$ lies at
the bottom of Complexity Zoology's inclusion hierarchy.

\begin{proposition}
If $\sC$ is a nontrivial, polynomial-time self-reducible complexity class, then
$\sP\subseteq\sC$.
\end{proposition}

\begin{proof}
Fix a nontrivial language $\cL\in\sC$, so that $\cL\neq\emptyset$ and
$\cL\in\Sigma^*$. Then there exists $x_1\in\cL$ and $x_0\notin\cL$.

Now suppose $\cL^\prime\in\sP$.. Then define $f:\Sigma^*\rightarrow\Sigma^*$ to
be
\[
f(x)=\begin{cases}x_1&\text{ if }x\in\cL^\prime, \\
x_0&\text{ if }x\notin\cL^\prime.\end{cases}
\]
We have $f\in\FP$, since it can be determined whether or not $x\in\cL^\prime$ in
polynomial-time, and then writing $x_1$ or $x_0$ can be accomplished in constant
time. Therefore $\cL^\prime=f^{-1}[\cL]\in\sC$ by polynomial-time
self-reducibility, so we can conclude that $\sP\subseteq\sC$.
\end{proof}

Additionally, complexity classes should be closed under joins and projections.
The \textit{join} of a pair of languages $\cL,\cL^\prime\subseteq\Sigma^*$ is
\[
\cL\oplus\cL^\prime=\{x\in\Sigma^*:(x=0y\AND y\in\cL)\OR(x=1y\AND
y\in\cL^\prime)\}.
\]
The \textit{0-projection} of a language $\cL$ is
\[
\{x\in\Sigma^*:0x\in\cL\},
\]
and the \textit{1-projection} is defined similarly.
\section{Relations and Inclusions}

\begin{proposition}
  The $\id$, $\co$, and $\cocap$ operators satisfy the following properties:
  \begin{enumerate}
  \item $\cocap\leq\co$ and $\cocap\leq\id$;
  \item $\co$ is involutive, so that $\co\cdot\co=\id$;
  \item $\co\cdot\cocap=\cocap\cdot\co=\cocap$.
  \end{enumerate}
\end{proposition}

\begin{proof}
(1) and (2) are immediate from the definitions of the operators. For (3), we
have
\begin{align*}
\cocap\cdot\co\cdot\sC&=(\co\cdot\sC)\cap(\co\cdot\co\cdot\sC) \\
                      &=(\co\cdot\sC)\cap\sC \\
                      &=\cocap\cdot\sC,
\end{align*}
and
\begin{align*}
\cL\in\co\cdot\cocap\cdot\sC
&\Longleftrightarrow\Sigma^*\setminus\cL\in\cocap\cdot\sC \\
&\Longleftrightarrow\Sigma^*\setminus\cL\in\sC
\AND\Sigma^*\setminus\cL\in\co\cdot\sC \\
&\Longleftrightarrow\cL\in\co\cdot\sC\AND\cL\in\co\cdot\co\cdot\sC \\
&\Longleftrightarrow\cL\in\co\cdot\sC\AND\cL\in\sC \\
&\Longleftrightarrow\cL\in\cocap\cdot\sC.
\end{align*}
\end{proof}

For many operators, it is the case that $\sC\in\op\cdot\sC$ for every $\sC$,
because the definitions of these classes include an additional certificate or
advice string.

\begin{proposition}
$\id\subseteq\op$, where $\op=\poly,\oplus,\BP,\sP,\sN\text{ or }\exppad$.
\end{proposition}

\begin{proof}
Fix $\cL\in\sC$. Then $\cL\in\op\cdot\sC$ for each possible choice of $\op$:
\begin{itemize}
\item If $\op=\poly$, take $\cL^\prime=\cL$ and $p(n)=\epsilon$ for all $n\in\N$
  in the definition of $\poly\cdot\sC$.
\item If $\op=\oplus,\BP,\sP,\text{ or }\sN$, take $\cL^\prime=\cL$ and
  $p(n)=0$ for all $n\in\N$ in the definition of $\op\cdot\sC$.
\item If $\op=\exppad$, take $\cL^\prime=\cL$ and $f(n)=\epsilon$ for all
  $n\in\N$ in the definition of $\exppad\cdot\sC$.
\end{itemize}
\end{proof}
Since the condition $\cL\in\BP\cdot\sC$ is a strengthening of the condition that
$\cL\in\sP\cdot\sC$, the following is immediate.
\begin{proposition}
$\BP\leq\sP$.
\end{proposition}

We next consider some commutativity properties.

\begin{proposition}
$\co\cdot\op=\op\cdot\co$, where $\op=\BP,\sP,\text{ or }\poly$.
\end{proposition}

\begin{proof}
For each of the possible choices of $\op$, the definition of $\op\cdot\sC$ has
the following form:
\[
op\cdot\sC:=\{\cL\subseteq\Sigma^*:(\exists\cL^\prime\in\sC)
\psi(\cL,\cL^\prime)\},
\]
where $\psi(\cL,\cL^\prime)$ is a proposition having the property that
\[
\psi(\cL,\Sigma^*\setminus\cL^\prime)\Longleftrightarrow
\psi(\Sigma^*\setminus\cL,\cL^\prime).
\]
Thus,
\begin{align*}
\co\cdot\op\cdot\sC
&=\{\cL\subseteq\Sigma^*:(\exists\cL^\prime\in\sC)\psi(\Sigma^*\setminus\cL,\cL^\prime)\} \\
&=\{\cL\subseteq\Sigma^*:(\exists\cL^\prime\in\sC)
\psi(\cL,\Sigma^*\setminus\cL^\prime)\} \\
&=\{\cL\subseteq\Sigma^*:(\exists\cL^\prime\in\co\cdot\sC)
\psi(\cL,\cL^\prime)\} \\
&=\op\cdot\co\cdot\sC
\end{align*}
for each possible choice of $\op$.
\end{proof}
A similar argument, based on the structure of te definitions of the relevant
operators, can be used to show that $\poly$ commutes with $\oplus$, $\sN$, and
$\sP$.

\begin{proposition}
If complexity classes are assumed to be polynomial-time self-reducible, then
$\poly\cdot\op=\op\cdot\poly$, where $\op=\oplus,\sN,\text{ or }\poly$.
\end{proposition}

\begin{proof}
We say that $\cL\in\poly\cdot\op\cdot\sC$ if there exist $\cL^\prime\in\sC$,
$p\in\Oh(n*)$, and $q\in|\Oh(n^*)|$ such that for every $x\in\Sigma^*$,
\[
x\in\cL\Longleftrightarrow
(Qy\in\Sigma^{p(|\inner{x,q(|x|)}|)})[\inner{\inner{x,q(|x|)},y}\in\cL^\prime],
\]
where $Q$ is the quantifier in the definition of the operator that is being
considered. Similarly, we say that $\cL\in\op\cdot\poly\cdot\sC$ if there exist
$\cL^\prime\in\sC$, $p\in\Oh(n*)$, and $q\in|\Oh(n^*)|$ such that for every
$x\in\Sigma^*$,
\[
x\in\cL\Longleftrightarrow
(Qy\in\Sigma^{p(|x|)})[\inner{\inner{x,y},q(|\inner{x,y}|)}\in\cL^\prime].
\]
The condition $\cL\in\poly\cdot\op\cdot\sC$ is equivalent to the condition that
there are $\cL^\prime\in\sC$, $\bar p\in\Oh(n^*)$, and $q\in|\Oh(n^*)|$ such
that for all $x\in\Sigma^*$,
\[
x\in\cL\Longleftrightarrow
(Qy\in\Sigma^{\bar p(|x|)})[\inner{\inner{x,q(|x|)},y}\in\cL^\prime].
\]
For instance, if $\cL\in\poly\cdot\op\cdot\sC$, then we can set $\bar
p(n)=p(N)$, where $N=|\inner{x,q(|x|)}|$ for $|x|=n$. Likewise, in the
conditions for $\cL\in\op\cdot\poly\cdot\sC$ we can replace $q$ with a $\bar q$
so that
\[
x\in\cL\Longleftrightarrow
(Qy\in\Sigma^{p(|x|)})[\inner{\inner{x,y},\bar q(|x|)}\in\cL^\prime].
\]
The rewritten conditions for $\cL\in\poly\cdot\op\cdot\sC$ and
$\cL\in\op\cdot\poly\cdot\sC$ are then equivalent to each other because a
mapping between $\inner{\inner{x,z},y}$ and $\inner{\inner{x,y},z}$ is
polynomial-time computable.
\end{proof}

\begin{proposition}
If complexity classes are assumed to be polynomial-time self-reducible, then
$\co\cdot\oplus=\oplus\cdot\co$.
\end{proposition}

Finally, we consider the properties of the operator $\sC\mapsto\sP^\sC$.
\begin{proposition}
For any complexity class $\sC$,
\begin{enumerate}
\item $\sC\subseteq\sP^\sC$;
\item $\co\cdot\sC\subseteq\sP^\sC$;
\item $\co\cdot\sP^\sC=\sP^{\co\cdot\sC}=\sP^\sC$.
\end{enumerate}
\end{proposition}
\begin{proof}
If $\cL\in\sC$ and $A$ is the decision function for $\cL$, then
$\cL\subseteq\sP^A\subseteq\sP^\sC$. Hence $\sC\subseteq\sP^\sC$. Moreover,
$\sP^A$ is a symmetric class for every $A$, and so
\begin{align*}
\cL\in\co\cdot\sP^C
&\Longleftrightarrow\Sigma^*\setminus\cL\in\sP^\sC \\
&\Longleftrightarrow(\exists A\in\sC)[\Sigma^*\setminus\cL\in\sP^A] \\
&\Longleftrightarrow(\exists A\in\sC)[\cL\in\sP^A] \\
&\Longleftrightarrow\cL\in\sP^\sC,
\end{align*}
and
\begin{align*}
\cL\in\sP^{\co\cdot\sC}
&\Longleftrightarrow(\exists A\in\sC)[\cL\in\sP^{1-A}] \\
&\Longleftrightarrow(\exists A\in\sC)[\cL\in\sP^A] \\
&\Longleftrightarrow\cL\in\sP^\sC,
\end{align*}
because $\sP^A=\sP^{1-A}$ for every oracle $A$. Thus, (1) and (3) are true. (2)
then follows immediately, because
$\sC\subseteq\sP^\sC\Longrightarrow\co\cdot\sC
\subseteq\co\cdot\sP^\sC\Longrightarrow\co\cdot\sC\subseteq\sP^\sC$.
\end{proof}

\begin{proposition}
For any complexity class non-trivial, polynomial-time self reducible class
$\sC$ that is closed under joins, $\poly\cdot\sP^\sC=\sP^{\poly\cdot\sC}$ and
$\BP\cdot\sP^\sC=\sP^{\BP\cdot\sC}$.
\end{proposition}

\begin{proof}[Proof of first equation]
First, we show that it is unconditionally the case that
$\sP^{\poly\cdot\sC}\subseteq\poly\cdot\sP^\sC$. Suppose that
$\cL\in\sP^{\poly\cdot\sC}$. Then there is a polynomial-time algorithm with
$A$-oracle that computes $\cL$, where $A$ is a decision function for a language
$\cL^\prime\in\poly\cdot\sC$. By the definition of the $\poly$ operator, there
exists a language $\cL^{\prime\prime}\in\sC$ and an advice function
$p\in|\Oh(n^*)|$ such that $x\in\cL^\prime$ if and only if
$\inner{x,p(|x|)}\in\cL^{\prime\prime}$.

Let $A^\prime$ indicate the decision function for $\cL^{\prime\prime}$. Also,
let $f\in\Oh(n^*)$ denote time bound for the $\sP^A$ algorithm for $\cL$, so
that the question of whether $x\in\cL$ is decided in at most $f(|x|)$
computational steps. Define $P:\N\rightarrow\Sigma^*$ so that, for each
$n\in\N$, $P(n)$ is the concatenation $p(0)p(1)\ldots p(f(n))$.

The following algorithm in $\poly\cdot\sP^{A^\prime}$ decides whether $x\in\cL$:
\begin{enumerate}
\item The advice function is $P$. Note that $P\in|\Oh(n^*)|$, because for
  sufficiently large $n$ $|P(n)|$ is at most $f(n)|p(f(n))|$.
\item Follow the $\sP^A$ algorithm for $\cL$ exactly, except when there is an
  oracle call.
\item When an oracle call to $A$ occurs with the string $y\in\Sigma^*$, replace
  it with an oracle call to $A^\prime$ with the string $\inner{y,p(|y|)}$. This
  oracle call is possible because $P(|x|)$ contains the advice strings for all
  $y$ that are short enough for the $\sP^A$ algorithm to be able to query the
  oracle.
\end{enumerate}
Thus, we have $\cL\in\poly\cdot\sP^{A^\prime}\subseteq\poly\cdot\sP^\sC$, and we
can conclude that $\sP^{\poly\cdot\sC}\subseteq\poly\cdot\sP^\sC$ in all cases.

For the inclusion $\poly\cdot\sP^\sC\subseteq\sP^{\poly\cdot\sC}$, suppose that
$\cL\in\poly\cdot\sP^\sC$. This means that there exists a $P^A$ algorithm that
decides $\cL$ when provided with some advice function $p\in|\Oh(n^*)|$, where
$A$ is the decision function of some $\cL^\prime\Sigma^*\rightarrow\Sigma$
according to the following rules:
\begin{itemize}
\item If $x=0\inner{y,z}$, then $A^\prime(x)$ is equal to the zth bit of
  $p(|y|)$.
\item If $x=1y$, then $A^\prime(x)=A(y)$.
\end{itemize}
The language $\cL^{\prime\prime}$ determined by $A^\prime$ is the join of two
languages. One, which we will call $\cL_0^{\prime\prime}$, is the set of all
$\inner{y,z}$ such that the $z$th bit of $p(|y|)$ is 1; the second language is
$\cL^\prime$.

We claim that $\cL^{\prime\prime}$ lies in $\poly\cdot\sC$. To prove this, it is
enough to show that $\cL_1^{\prime\prime}\in\sP/\poly$. Then
$\sP/\poly\subseteq\poly\cdot\sC$ (we know that $\sP\subseteq\sC$ because $\sC$
si assumed to be nontrivial and polynomial-time self-reducible), and
$\cL^\prime\in\poly\cdot\sC$ by assumption, so
$\cL^{\prime\prime}\in\poly\cdot\sC$ by the hypothesis that $\sC$, and therefore
$\poly\cdot\sC$, is closed under joints.

To see that $\cL_1^{\prime\prime}\in\sP/\poly$, let $P\in|\Oh(n^*)|$ be the
function defined by the concatenation $P(n)=p(0)p(1)\ldots p(n)$. Then, given
$\inner{y,z}$, $P$ can be used as an advie function to check whether the $z$th
bit of $p(|y|)$ is 1.

Hence $\cL^{\prime\prime}\in\poly\sC$. To show that $\cL\in\sP^{\poly\cdot\sC}$,
we show that $\cL\in P^{A^\prime}$. The following is a $\sP^{A^\prime}$
algorithm for deciding whether $x\in\cL$:
\begin{enumerate}
\item First, extract the advice string $p(|x|)$ from the $A^\prime$-oracle. Make
  the oracle queries $0\inner{x,j}$, $j\leq|p(|x|)|$ until the entirety of
  $p(|x|)$ has been recorded.
\item Carry out the rest of the computation according to the $\poly\cdot\sP^A$
  algorithm. Replace oracle queries to $A$ about the string $y$ with oracle
  queries to $A^\prime$ about the string $y$.
\end{enumerate}
Therefore $\cL\in\sP^{\poly\cdot\sC}$, concluding the proof that $\sP^{\poly\cdot\sC}=\poly\cdot\sP^\sC$.
\end{proof}

The first of these equations is true because advice can be moved between the
polynomial-time computation and the oracle class, while the latter equation is
true because randomness in computation is equivalent to randomness in the
oracle. The class $\sC$ must satisfy the regularity properties discussed in the
previous section.

\end{document}
