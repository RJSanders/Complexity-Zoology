\documentclass[12pt]{amsart}

\usepackage[margin=1in]{geometry}
\usepackage{lmodern}
\usepackage{amsmath,amsthm,amssymb,mathrsfs}

\title{To-do List Inference Algorithm}
\date{}
\author{Robert Sanders}

\newtheorem*{theorem}{Theorem}
\newtheorem*{lemma}{Lemma}
\newtheorem*{proposition}{Proposition}
\newtheorem*{corollary}{Corollary}

\theoremstyle{definition}
\newtheorem*{definition}{Definition}

\theoremstyle{remark}
\newtheorem*{claim}{Claim}
\newtheorem*{remark}{Remark}

\newcommand{\C}{\mathbb{C}}
\newcommand{\N}{\mathbb{N}}
\newcommand{\Q}{\mathbb{Q}}
\newcommand{\R}{\mathbb{R}}
\newcommand{\Z}{\mathbb{Z}}
\newcommand{\ra}{\rightarrow}
\newcommand{\cL}{\mathcal{L}}
\newcommand{\start}{\mathtt{start}}
\newcommand{\halt}{\mathtt{halt}}
\newcommand{\Left}{\mathsf{L}}
\newcommand{\Stay}{\mathsf{S}}
\newcommand{\Right}{\mathsf{R}}
\newcommand{\A}{\mathsf{A}}
\newcommand{\sC}{\mathsf{C}}
\newcommand{\co}{\mathsf{co}}
\newcommand{\cocap}{\mathsf{cocap}}
\newcommand{\NP}{\mathsf{NP}}
\newcommand{\sP}{\mathsf{P}}
\newcommand{\poly}{\mathsf{poly}}
\newcommand{\msf}[1]{\mathsf{#1}}
\newcommand{\id}{\msf{id}}
\newcommand{\AND}{\mathbin{\&}}
\newcommand{\OR}{\mathbin{\text{or}}}
\newcommand{\Oh}{\mathcal{O}}
\newcommand{\inner}[1]{\left\langle#1\right\rangle}
\newcommand{\op}{\msf{op}}
\newcommand{\sN}{\msf{N}}
\newcommand{\BP}{\msf{BP}}
\newcommand{\exppad}{\msf{exppad}}
\newcommand{\FP}{\msf{FP}}
\newcommand{\CZ}{\msf{CZ}}

\begin{document}
\maketitle

Both the primary inference engine and the propogation of operator partial
functions follow the same basic procedure. We begin with some database $D_0$,
which can be taken to be a set of propositions that are regarded as true. We
also assume that there is a set $R$ of inference rulses, which are tuples of the
form
\[
\varphi_1,\varphi_2,\ldots,\varphi_n\vdash\psi.
\]
The inference algorithm that Complexity Zoology employs is as follows:
\begin{enumerate}
\item Populate a list $L$ with the proopositions in $D_0$, and set
  $D=\emptyset$.
\item While $L$ is nonempty, carry out the steps (3) through (6).
\item Remove the top proposition $\varphi$ from $L$.
\item If $\varphi\in D$, return to step (3).
\item Add $\varphi$ to the set $D$.
\item For each inference rule $\varphi_1,\varphi_2,\ldots,\varphi_n\vdash\psi$
  and all $\varphi_1^\prime,\varphi_2^\prime,\ldots,\varphi_{n-1}^\prime\in D$,
  check whether some permutation of
  $\varphi,\varphi_1^\prime,\ldots,\varphi_{n-1}^\prime$ matches
  $\varphi_1,\varphi_2,\ldots,\varphi_n$. If it does, append $\varphi$ to $L$.
\end{enumerate}
The resulting database $D$ has $D_0$ as a subset and is closed under inference
rules. Moreover, this algorith eventually terminates, because when a proposition
has been removed from $L$ once, it cannot again result in any additional
proposition being appended to $L$. Thus, the algorithm deduces all logical
consequences of the initial database $D_0$, and it does so faster than the naive
approach of repeatedly applying all inference rules to all the propositions in
$D$ until $D$ grows no further.

\end{document}