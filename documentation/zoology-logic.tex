\documentclass[12pt]{amsart}

\usepackage[margin=1in]{geometry}
\usepackage{lmodern}
\usepackage{amsmath,amsthm,amssymb,mathrsfs}

\title{The Logic of Complexity Zoology}
\date{}
\author{Robert Sanders}

\newtheorem*{theorem}{Theorem}
\newtheorem*{lemma}{Lemma}
\newtheorem*{proposition}{Proposition}
\newtheorem*{corollary}{Corollary}

\theoremstyle{definition}
\newtheorem*{definition}{Definition}

\theoremstyle{remark}
\newtheorem*{claim}{Claim}
\newtheorem*{remark}{Remark}

\newcommand{\C}{\mathbb{C}}
\newcommand{\N}{\mathbb{N}}
\newcommand{\Q}{\mathbb{Q}}
\newcommand{\R}{\mathbb{R}}
\newcommand{\Z}{\mathbb{Z}}
\newcommand{\ra}{\rightarrow}
\newcommand{\cL}{\mathcal{L}}
\newcommand{\start}{\mathtt{start}}
\newcommand{\halt}{\mathtt{halt}}
\newcommand{\Left}{\mathsf{L}}
\newcommand{\Stay}{\mathsf{S}}
\newcommand{\Right}{\mathsf{R}}
\newcommand{\A}{\mathsf{A}}
\newcommand{\sC}{\mathsf{C}}
\newcommand{\co}{\mathsf{co}}
\newcommand{\cocap}{\mathsf{cocap}}
\newcommand{\NP}{\mathsf{NP}}
\newcommand{\sP}{\mathsf{P}}
\newcommand{\poly}{\mathsf{poly}}
\newcommand{\msf}[1]{\mathsf{#1}}
\newcommand{\id}{\msf{id}}
\newcommand{\AND}{\mathbin{\&}}
\newcommand{\OR}{\mathbin{\text{or}}}
\newcommand{\Oh}{\mathcal{O}}
\newcommand{\inner}[1]{\left\langle#1\right\rangle}
\newcommand{\op}{\msf{op}}
\newcommand{\sN}{\msf{N}}
\newcommand{\BP}{\msf{BP}}
\newcommand{\exppad}{\msf{exppad}}
\newcommand{\FP}{\msf{FP}}
\newcommand{\CZ}{\msf{CZ}}

\begin{document}
\maketitle

Propositions in Complexity Zoology are inclusions of the form
$\sC_1\subseteq\sC_2$, where $\sC_1$ and $\sC_2$ are complexity classes. Each
such inclusion is true or false in a particular \textit{world} in the sense of
modal logic: for a world $W$, we write $\sC_1\subseteq_W\sC_2$ to indicate that
the inclusion is true in $W$. Each world $W$ has a transitive dual $W^*$ with
respect to which the following inference rules are true:
\begin{align*}
\sC_1\subseteq_W\sC_2\AND\sC_2\subseteq_{W^*}\sC_3
&\Longrightarrow\sC_1\subseteq_W\sC_3, \\
\sC_1\subseteq_{W^*}\sC_2\AND\sC_2\subseteq_W\sC_3
&\Longrightarrow\sC_1\subseteq_W\sC_3.
\end{align*}
Additionally, there is a partial ordering $\rightarrow$ on the set of worlds
such that if $W_1\rightarrow W_2$, then the following inference rule is true:
\[
\sC_1\subseteq_{W_1}\sC_2\Longrightarrow\sC_1\subseteq_{W_2}\sC_2.
\]
In the current version of Complexity Zoology, there are six worlds:
\begin{align*}
E &\longleftrightarrow\text{every oracle}, \\
A &\longleftrightarrow\text{every algebraic oracle}, \\
X &\longleftrightarrow\text{some algebraic oracle}, \\
R &\longleftrightarrow\text{the random oracle}, \\
T &\longleftrightarrow\text{the trivial oracle}, \\
O &\longleftrightarrow\text{some oracle}.
\end{align*}
For example, we write $\sC_1\subseteq_X\sC_2$ if $\sC_1^A\subseteq\sC_2^A$ for
some algebraic oracle $A$. The worlds $E$, $A$, $R$, and $T$ are all
\textit{transitive} worlds in the sense that they are their own transitive
duals: $E^*=E$, $A^*=A$, $R^*=R$, and $T^*=T$. On the other hand $X^*=A$ and
$O^*=E$.

The remaining inference rules pertain to complexity class operators. For each
operator $\op$ and world $W$,
\[
\sC_1\subseteq_W\sC_2\Longrightarrow\op\cdot\sC_1\subseteq_W\op\cdot\sC_2
\]
is an inference rule. There is also a special pair of inference rules involving
the $\co$ and $\cocap$ operators: for a complexity class $\sC$, $\cocap\cdot\sC$
is the \textit{meet} of $\sC$ and $\co\cdot\sC$. For each world $W$, we have
\begin{align*}
\sC_1\subseteq_W\sC_2\AND\sC_1\subseteq_{W^*}\sC_2
&\Longleftrightarrow\sC_1\subseteq_W\cocap\cdot\sC_2, \\
\sC_1\subseteq_{W^*}\sC_2\AND\sC_1\subseteq_W\sC_2
&\Longleftrightarrow\sC_1\subseteq_W\cocap\cdot\sC_2.
\end{align*}

This logical framework -- the propositions that specify inclusions, the worlds,
and the inference rules -- represents the basic knowledge and reasoning of which
Complexity Zoology is capable. However, since the purpose of Complexity Zoology
is to partially automate the process of surveying complexity theory, it requires
a formal system that can describe not merely whether a statement is true or
false, but whether it is regarded as proven, unproven, or open.

Each of the statements that we have described so far, along with the inference
rules, can be formally expressed in the language of set theory. We assume that
there is a set $P$ of inclusion statements and negations thereof that are
regarded as \textit{proven}. We assume that $P$ is closed under the logical
system of Complexity Zoology. In other words, if $CZ$ is the set of formalized
inference rules that Complexity Zoology uses, and if $CZ\cup P\vdash\varphi$
($\varphi$ is a logical consequence of $CZ\cup P$), where $\varphi$ is a
formalized inclusion of complexity classes, then $\varphi\in P$. Thus, we regard
$CZ$ as a system that is simple enough so that any of its implications from
proven facts should also be regarded as proven.

A formal inclusion $\varphi$ is regarded as \textit{proven} if it lies in $P$,
\textit{disproven} if its negation $\neg\varphi$ lies in $P$, and \textit{open}
if it is neither proven nor disproven. A statement is also sometimes called
\textit{provable} if it is not disproven and \textit{disprovable} if it is not
proven. We use the following notation:
\begin{align*}
P(\varphi) &\longleftrightarrow\varphi\text{ is proven}, \\
D(\varphi) &\longleftrightarrow\varphi\text{ is disproven}, \\
p(\varphi) &\longleftrightarrow\varphi\text{ is provable}, \\
d(\varphi) &\longleftrightarrow\varphi\text{ is disprovable}, \\
O(\varphi) &\longleftrightarrow\varphi\text{ is open}.
\end{align*}
The data input is a partial description of the set $P$ in the form of a list of
statements that are proven, disproven, or open. Complexity Zoology then attempts
to deduce as much as it can about which inclusions are proven, disproven, or
open. Complexity Zoology has no understanding of complexity theory as such; its
strength is in organizing the state of knowledge in the field -- that is,
Complexity Zoology is an expert at surveying complexity theory, not in
complexity theory itself.

Internally, Complexity Zoology represents propositions as a quadruple
\[
(\text{status}, \text{world}, \sC_1, \sC_2),
\]
where the status is either proven, disproven, provable, or disprovable; the
quadruple is interpreted as ``the inclusion $\sC_1\subseteq\sC_2$ has the
specified status in the specified world.'' Then, from the base inference rules
of the system $CZ$, we generate the full set of inference rules as follows:
\begin{enumerate}
\item For any formalized inclusion $\varphi$, $P(\varphi)\Longrightarrow
  p(\varphi)$ and $D(\varphi)\Longrightarrow d(\varphi)$.
\item If $\varphi_1\AND\cdots\AND\varphi_n\Longrightarrow\psi$ is an
  inference rule, then
  \[
  P(\varphi_1)\AND\cdots\AND P(\varphi_n)\Longrightarrow P(\psi).
  \]
\item The partial involutions $P\mapsto D\mapsto P$ and
  $P\mapsto d\mapsto P,D\mapsto p\mapsto D$ are implication-reversing. For
  example, if $P(\varphi)\Longrightarrow P(\psi)$, then
  $D(\psi)\Longrightarrow D(\varphi)$, and if
  \[
  P(\varphi_1)\AND\cdots\AND P(\varphi_n)\Longrightarrow P(\psi),
  \]
  then
  \[
  P(\varphi_1)\AND\cdots\AND d(\psi)\Longrightarrow d(\varphi_n).
  \]
\end{enumerate}
For instance, if $\sC_1^A\subseteq\sC_2^A$ for every oracle $A$ and
$\sC_1^A\not\subseteq\sC_3^A$ for some oracle $A$, then
$\sC_2^A\not\subseteq\sC_3^A$ for some oracle $A$. In our notation,
\[
P(\sC_1\subseteq_E\sC_2)\AND D(\sC_1\subseteq_E\sC_3)\Longrightarrow
D(\sC_2\subseteq_E\sC_3).
\]  

\end{document}