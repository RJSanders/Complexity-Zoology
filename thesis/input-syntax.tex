\section{Input Files and Syntax}

Complexity Zoology reads its initial data from two plain text files: one
consisting of complexity classes and their inclusions and oracle separations
(\texttt{classes.txt}), and another consisting of complexity class operators
and their relations (\texttt{operators.txt}). Both classes and operators must be
declared in their respective files before they can be used. If an undeclared
class or operator is used in some inclusion, separation, or relation, Complexity
Zoology will halt and print an error. A declaration consists of a line of text
having the following form:
\[
\texttt{NAME : description : keyword1, keyword2, keyword3}
\]
Here, \texttt{NAME} is the name by which the class or operator is referenced
both internally and in the output. Any alphanumeric characters, as well as
hyphens, can be used for names. The \texttt{description} is a short phrase
used to indicate the nature of the class or operator to a human reader; the
program itself does not use the description. Like names, descriptions should
consist of alphanumeric characters and hyphens, although whitespace is also
allowed. Finally, keywords can optionally be included in a declaration.
Keywords follow the same naming rules as class and operator names, and multiple
keywords must be separated by commas, which can be surrounded by any amount of
whitespace. If there are no keywords, the second colon must be omitted.

Keywords are used to provide additional information about the class or
operator being declared. Most often, a keyword is shorthand for commonly
arising relations. For example, the class keyword \texttt{symmetric} is
equivalent to including the line \texttt{C = co.C}, where \texttt{C} is the
name of the declared class. The following keywords are defined for complexity
classes:
\begin{itemize}
\item \texttt{hidden} -- The class is suppressed in the final output, but it is
  still included for the purposes of calculation and deduction.
\item \texttt{ignore} -- The class is not included in calculation or output;
  any relations involving the class are effectively commented out.
\item \texttt{preferred} -- If this class is equal to another, this class should
  be the preferred name.
\item \texttt{preferred[\#]} -- Here, the symbol \texttt{\#} should be replaced
  with a positive integer and indicates the \textit{preference rank} of the
  declared class. When Complexity Zoology chooses a name for equal complexity
  classes, it favors those with the smaller preference rank. The
  \texttt{preferred} keyword is equivalent to \texttt{preferred[1]}.
\item \texttt{symmetric} -- The class is \textit{symmetric} in the sense of
  being equivalent to its complement: for a complexity class $\sC$, this means
  that $\sC=\co\cdot\sC$ relative to every oracle.
\end{itemize}
For operators, there is currently only one keyword:
\begin{itemize}
\item \texttt{idempotent} -- Applying the operator to a class a second time has
  the same effect as applying it once. This keyword is equivalent to the
  relation \texttt{op.op = op}.
\end{itemize}

Aside from class and operator declarations, the input files also include
\textit{relations} describing what is (initially) known about the classes and
operators. For complexity classes, relations are either statements of equality,
statements of inclusion, or statements of oracle separation. Suppose that
$\sC_1$ and $\sC_2$ are classes declared with the names \texttt{C1} and
\texttt{C2}, respectively. Then we have these valid relations:
\begin{itemize}
\item \texttt{C1 = C2} $\leftrightarrow$ $\sC_1^A=\sC_2^A$ for every oracle 
  $A$.
\item \texttt{C1 < C2} $\leftrightarrow$ $\sC_1^A\subseteq\sC_2^A$ for every
  oracle $A$.
\item \texttt{C1 r< C2} $\leftrightarrow$ $\sC_1^A\subseteq\sC_2^A$ with 
  probability 1 for a random oracle $A$.
\item \texttt{C1 a< C2} $\leftrightarrow$ $\sC_1^A\subseteq\sC_2^A$ for 
  every algebraic oracle $A$.
\item \texttt{C1 t< C2} $\leftrightarrow$ $\sC_1\subseteq\sC_2$ relative to 
  the trivial oracle.
\item \texttt{C1 x< C2} $\leftrightarrow$ $\sC_1^A\subseteq\sC_2^A$ for some
  algebraic oracle $A$.
\item \texttt{C1 o< C2} $\leftrightarrow$ $\sC_1^A\subseteq\sC_2^A$ for some
  oracle $A$.
\item \texttt{C1 osep C2} $\leftrightarrow$ $\sC_1^A\not\subseteq\sC_2^A$ 
  for some oracle $A$.
\item \texttt{C1 rsep C2} $\leftrightarrow$ $\sC_1^A\not\subseteq\sC_2^A$ 
  with probability 1 for a random oracle $A$.
\item \texttt{C1 xsep C2} $\leftrightarrow$ $\sC_1^A\not\subseteq\sC_2^A$ 
  for some algebraic oracle $A$.
\item \texttt{C1 tsep C2} $\leftrightarrow$ $\sC_1\not\subseteq\sC_2$ 
  relative to the trivial oracle.
\item \texttt{C1 asep C2} $\leftrightarrow$ $\sC_1^A\not\subseteq\sC_2^A$ 
  for every algebraic oracle $A$.
\item \texttt{C1 sep C2} $\leftrightarrow$ $\sC_1^A\not\subseteq\sC_2^A$ for
  every oracle $A$.
\end{itemize}
For operators $\op_1$, $\op_2$, $\op_3$ and $\op_4$ with declared names
\texttt{op1}, \texttt{op1}$, \texttt{op1}$, and \texttt{op1},
respectively, we have these relations:
\begin{itemize}
\item \texttt{op1.op2 = op3.op4} $\leftrightarrow$ 
  $\op_1\cdot\op_2=\op_3\cdot\op_4$. Omitting
  one of the operators on either side of the equation is allowed; e.g.,
  \texttt{op1.op2 = op3}, which is interpreted as $\op_1\cdot\op_2=\op_3$.
  Complexity Zoology implements this by replacing the missing operator with the
  identity operator $\id$.
\item \texttt{op1 z= op2} $\leftrightarrow$ $\op_1$ and $\op_2$ commute. 
  This is equivalent to including the line \texttt{op1.op2 = op2.op1}.
\item \texttt{op1 p= op2} $\leftrightarrow$ $\op_2$ absorbs $\op_1$ on the 
  left and right. This is equivalent to including the lines \texttt{op1.op2 = 
  op2} and \texttt{op2.op1 = op2}.
\item \texttt{op1 < op2} $\leftrightarrow$ $\op_1\leq\op_2$; i.e.,
  $\op_1\cdot\sC\subseteq\op_2\cdot\sC$ for every complexity class $\sC$.
\end{itemize}
For all relations, with the exception of the quadratic operator relations of the
form \texttt{op1.op2 = op3.op4}, it is possible to include multiple classes or
operators separated by commas:
\[
\texttt{C1,C2 = C3,C4,C5}
\]
This example is equivalent to including six lines: \texttt{C1 = C3},
\texttt{C1 = C4}, \texttt{C1 = C5}, \texttt{C2 = C3}, \texttt{C2 = C4}, and
\texttt{C2 = C5}.

Lastly, there are two ways to include text that Complexity Zoology will ignore:
\textit{comments} and \textit{citations}. Comments consist of the character
\texttt{\#} followed by all text up to the end of the current line. Citations
consist of text surrounded by square brackets. In the current version of the
input files, citations are used both to refer to this documentation's
bibliography and to annotate common arguments.

