\section{Operator Propogation and Inference}

Operators in Complexity Zoology are implemented as partial funtions on the set
of all distinct complexity classes. Processing of operators occurs after the
processing of equality statements, so we can assume that we have a quotient map
$q:\mathcal{N}\rightarrow\cC$, where $\mathcal{N}$ is the set of all names for complexity
classes, $\cC$ is the set of all distinct complexity classes, and $q(x)=q(y)$ if
and only if $x$ and $y$ are names for the same class. The system's understanding
of a complexity class operator $\op$ is a partial function
$op:\cC\rightharpoonup\cC$; i.e., a function whose domain is a subset of $\cC$
that takes values in $\cC$. The function that defines $\op$ is necessarily a
partial one, because for a given complexity class $\sC$, the class $\op\cdot\sC$
might not be declared in Complexity Zoology's data set.

Initially, we assume that the following is true of our operator partial
functions:
\begin{enumerate}
\item $\id(x)=x$ for all $x\in\cC$.
\item If a class $x$ has a name of the form $\mathtt{op.y}$, where $y\in\mathcal{N}$ and
  $\mathtt{op}$ is the name of an operator, then we set $op(q(y))=x$.
\end{enumerate}
Then, the partial functions are expanded according to the quadratic relations
specified in the input file. More specifically, suppose that one such relation
is
\[
\op_1\cdot\op_2=\op_3\cdot\op_4.
\]
Then, the operator partial functions can be expanded according to the following
rules:
\begin{enumerate}
\item If $op_1(op_2(x))$ and $y=op_4(x)$ are defined, then define
  $op_3(y):=op_1(op_2(x))$.
\item If $op_3(op_4(x))$ and $y=op_2(x)$ are defined, then define
  $op_1(y):=op_3(op_4(x))$.
\end{enumerate}
These rules are applied iteratively until the partial functions can be expanded
no further. This process uses the same task list-based system that is used in
the primary inference engine (see Section \ref{todo}).

While the partial functions are expanding, it is possible that Complexity Zoology 
learns that $op(x)=y$ and $op(x)=z$, where $y$ and $z$ are different names for 
complexity classes. If this occurs, then Complexity Zoology stops and produces an 
error, because it is expected that all equalities between complexity classes are 
learned through explicit statements in the data file and the transitivity of the 
equality relation.