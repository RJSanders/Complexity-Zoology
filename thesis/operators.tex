\section{Complexity Class Operators}\label{operator-section}

This section, and the theory of complexity class operators generally, is based 
on a paper by Zachos and Pagourtzis \cite{zachos2003combinatory}.

A \textit{complexity class operator} $\op$ is an inclusion-preserving automorphism 
on the set of all complexity classes, written as $\op\cdot\sC$ for a complexity 
class $\sC$. Thus, if $\sC\subseteq\msf{D}$, then 
$\op\cdot\sC\subseteq\op\cdot\msf{D}$. Complexity Zoology's knowledge of operators 
consists of inequalities of the form $\op_1\leq\op_2$ and quadratic relations of 
the form $\op_1\cdot\op_2=\op_3\cdot\op_4$.
\begin{definition}
For complexity class operators $\op_1$, $\op_2$, $\op_3$, and $\op_4$, we denote by
$\op_1\leq\op_2$  the proposition that 
$(\op_1\cdot\sC)^f\subseteq(\op_2\cdot\sC)^f$ for each class $\sC$ and oracle $f$, 
and we denote by $\op_1\cdot\op_2=\op_3\cdot\op_4$ the proposition that 
$(\op_1\cdot\op_2\cdot\sC)^f=(\op_3\cdot\op_4\cdot\sC)^f$ for each class $\sC$ and 
oracle $f$.
\end{definition}

\subsection{Definitions}

The definitions of the following complexity
class operators preserve relativization. In other words, if an operator
$\op$ is defined by the property that
\[
  \op\cdot\sC=\{\cL\subseteq\Sigma^*:\varphi(\cL,\sC)\}
\]
for any complexity class $\sC$, then the relativized version of
$\op\cdot\sC$ is
\[
  (\op\cdot\sC)^f=\{\cL\subseteq\Sigma^*:\varphi(\cL,\sC^f)\}.
\]

The simplest operators are $\id$, the identity operator; $\cco$, which swaps
``yes'' and ``no'' answers to each decision problem; and $\cocap$, which takes
the intersection of a class with its complement.

\begin{definition} For each complexity class $\sC$, we set
\begin{align*}
  \id\cdot\sC&:=\{\cL\subseteq\Sigma^*:\cL\in\sC\}=\sC, \\
  \cco\cdot\sC&:=\{\cL\subseteq\Sigma^*:\Sigma^*\setminus\cL\in\sC\}, \\
  \cocap\cdot\sC&:=\{\cL\subseteq\Sigma^*:\cL\in\sC\AND\cL\in\cco\cdot\sC\}=
                  \sC\cap(\cco\cdot\sC).
\end{align*}
A class is \textbf{symmetric} if $\sC=\cco\cdot\sC$ with respect to every oracle.
\end{definition}
For example, the class $\sP$ is symmetric, while $\NP$ is not, because there is
an oracle $f$ relative to which $\NP^f\neq\co\NP^f$ (although, of course, this
is an open problem in the absence of an oracle).

The $\cpoly$ operator adds a polynomial-length \textit{advice string} to each 
input.
\begin{definition}
For a complexity class $\sC$, we define
\[
  \cpoly\cdot\sC=\{\cL\subseteq\Sigma^*:(\exists \cL^\prime\in\sC,
  |p(n)|=\Oh(n^*))(\forall x\in\Sigma^*)[x\in\cL\Longleftrightarrow
  \inner{x,p(|x|)}\in\cL^\prime]\}.
\]
\end{definition}
We allow advice functions to map to the null string $\epsilon$ of length
zero. In the case of tuples, $\inner{x,\epsilon}$ should be understood to be
$x$, so that, as we will see, $\cpoly\cdot\sC$ always contains $\sC$. For most
classes with polynomial advice, we write $\cpoly\cdot\sC=\sC/\poly$; we have, for
instance, $\sP/\poly$, $\NP/\poly$, and $\msf{BQP}/\poly$. We also use the suffixes
$/\msf{mpoly}$ and $/\msf{qpoly}$, meaning \textit{Merlinized polynomial advice} 
and \textit{quantum polynomial advice}, respectively, but neither of these can be 
rigorously defined as an operator. Merlinized polynomial advice is used to exempt a
probabilistic complexity class from satisfying a normally required probability gap 
when the advice string is unhelpful. Quantum polynomial advice consists of a string
of qubits rather than classical bits, and the qubits can be in any state. See 
Subsection \ref{class-definitions} for an 
example of a class with quantum polynomial advice and Subsection \ref{bqp-mpoly} 
for an explanation of Merlinized polynomial advice.

The operators $\oplus$, $\cN$, and $\cP$ are all defined in terms of
certificates, strings whose lengths are polynomials of the length of the
original input. For a quantifier $Q$, we can define an operator $\op_Q$ by
\[
  \op_Q\cdot\sC:=\{\cL\subseteq\Sigma^*:
  (\exists\cL^\prime\in\sC, p(n)=\Oh(n^*))(\forall x\in\Sigma^*)
  [x\in\cL\Longleftrightarrow(Qy\in\Sigma^{p(|x|)})[\inner{x,y}\in\cL^\prime]]\}.
\]
The aforementioned operators are then equal to $\op_Q$ for different choices of
$Q$.
\begin{definition}
The operators $\oplus$, $\cN$, and $\cP$ are defined as follows for a
complexity class $\sC$:
\begin{itemize}
\item $\oplus\cdot\sC:=\op_Q\cdot\sC$, where $(Qy\in S)$ means ``for an odd
  number of $y\in S$.''
\item $\cN\cdot\sC:=\op_Q\cdot\sC$, where $(Qy\in S)$ means $(\exists y\in S)$.
\item $\cP\cdot\sC:=\op_Q\cdot\sC$, where $(Qy\in S)$ means ``for at least 1/2
  of all $y\in S$.''
\end{itemize}
$\oplus$ is read as ``parity.''
\end{definition}

The bounded probabilistic operator $\cBP$ is defined similarly.
\begin{definition}
For each complexity class $\sC$,
\begin{align*}
\cBP\cdot\sC:=\{\cL\subseteq\Sigma^*:&(\exists\cL^\prime,p(n)=\Oh(n^*))(\forall
x\in\Sigma^*)[[x\in\cL\Longrightarrow
(\exists_{>2/3}\hspace{2pt}y\in\Sigma^{p(|x|)})[\inner{x,y}\in\cL^\prime]] \\
&\AND[x\notin\cL\Longrightarrow(\exists_{>2/3}\hspace{2pt}y\in\Sigma^{p(|x|)})
[\inner{x,y}\notin\cL^\prime]]]\},
\end{align*}
where $(\exists_{>2/3}\hspace{2pt}y\in\Sigma^{p(|x|)})$ is understood to
mean ``for more than 2/3 of all $y\in\Sigma^{p(|x|)}$.''
\end{definition}
All of these operators are named in such a way that they suggest the definitions
of common complexity classes; for example, $\NP=\cN\cdot\sP$,
$\sP\sP=\cP\cdot\sP$, $\msf{BPP}=\cBP\cdot\sP$, and $\oplus\sP=\oplus\cdot\sP$.

Also, we have $\exppad$, which adds an exponential length of
zeros to input, generally for the purpose of buying additional computational
time.
\begin{definition}
Write $f=\Oh(2^{poly})$ if $f(n)=\Oh(2^{p(n)})$ for some $p(n)=\Oh(n^*)$. Then,
for a complexity class $\sC$,
\[
\exppad\cdot\sC:=\{\cL\subseteq\Sigma^*:(\exists\cL^\prime\in\sC,f=
\Oh(2^{poly}))[x\in\cL\Longleftrightarrow x0^{f(|x|)}\in\cL^\prime]\}.
\]
\end{definition}
$\msf{NEXP}$, for example, is not defined to  be $\cN\cdot\msf{EXP}$, but rather
$\exppad\cdot\NP$.

Finally, note that for any fixed complexity class $\sC$, the map 
$\sC^\prime\mapsto\sC^{\sC^\prime}$ defines an operator. In Complexity Zoology, 
$\sC\mapsto\sP^\sC$, where $\sP$ is the class of polynomial-time computable 
languages, is a declared operator.

\subsection{Properties of Complexity Classes}

Proving the properties of complexity operators often requires that the
underlying complexity classes themselves have certain regularity
properties. First, every complexity class of interest should be
\textit{nontrivial} in the sense that it contains a nonempty language not equal
to $\Sigma^*$. We also expect that if $\cL\in\sC$, then any languages that are
reducible to $\cL$ in polynomial time are also in $\sC$.

\begin{definition}
A complexity class $\sC$ is \textbf{closed under polynomial-time reductions} if for
every
$\cL\in\sC$ and every function $f\in\FP$,
\[
f^{-1}[\cL]=\{x\in\Sigma^*:f(x)\in\cL\}\in\sC.
\]
\end{definition}

Every class in this project is relativizingly nontrivial and polynomial-time
self-reducible, so that for each oracle $g$, if $f\in\FP^g$ and $\cL\in\sC^g$
then $f^{-1}[\cL]\in\sC^g$. As a result, $\sP$ lies at
the bottom of Complexity Zoology's inclusion hierarchy.

\begin{proposition}
If $\sC$ is a nontrivial complexity class that is closed under polynomial-time 
reductions, then
$\sP\subseteq\sC$.
\end{proposition}

\begin{proof}
Fix a nontrivial language $\cL\in\sC$, so that $\cL\neq\emptyset$ and
$\cL\in\Sigma^*$. Then there exists $x_1\in\cL$ and $x_0\notin\cL$.

Now suppose $\cL^\prime\in\sP$. Then define $f:\Sigma^*\rightarrow\Sigma^*$ to
be
\[
f(x)=\begin{cases}x_1&\text{ if }x\in\cL^\prime, \\
x_0&\text{ if }x\notin\cL^\prime.\end{cases}
\]
We have $f\in\FP$, since it can be determined whether or not $x\in\cL^\prime$ in
polynomial time, and then writing $x_1$ or $x_0$ can be accomplished in constant
time. Therefore $\cL^\prime=f^{-1}[\cL]\in\sC$ by polynomial-time
self-reducibility, so we can conclude that $\sP\subseteq\sC$.
\end{proof}

Additionally, complexity classes should be closed under joins, projections, and 
polynomial majorities. The \textit{join} of a pair of languages 
$\cL,\cL^\prime\subseteq\Sigma^*$ is
\[
\cL\oplus\cL^\prime=\{x\in\Sigma^*:(x=0y\AND y\in\cL)\OR(x=1y\AND
y\in\cL^\prime)\}.
\]
The \textit{0-projection} of a language $\cL$ is
\[
\{x\in\Sigma^*:0x\in\cL\},
\]
and the \textit{1-projection} is defined similarly. Given a complexity class $\sC$,
a language $\cL$ is a \textit{polynomial majority} of $\sC$ if there exist 
$\cL^\prime\in\sC$, $p(n)=\Oh(n^*)$ such that for each $x\in\Sigma^*$, $x\in\cL$ if 
and only if $\inner{x,m}\in\cL^\prime$ for a majority of $m\in\{0,\ldots,p(|x|)\}$.

\subsection{Relations and Inclusions}

\begin{proposition}
  The $\id$, $\cco$, and $\cocap$ operators satisfy the following properties:
  \begin{enumerate}
  \item $\cocap\leq\cco$ and $\cocap\leq\id$;
  \item $\cco$ is involutive, so that $\cco\cdot\cco=\id$;
  \item $\cco\cdot\cocap=\cocap\cdot\cco=\cocap$.
  \end{enumerate}
\end{proposition}

\begin{proof}
(1) and (2) are immediate from the definitions of the operators. For (3), we
have
\begin{align*}
\cocap\cdot\cco\cdot\sC&=(\cco\cdot\sC)\cap(\cco\cdot\cco\cdot\sC) \\
                      &=(\cco\cdot\sC)\cap\sC \\
                      &=\cocap\cdot\sC,
\end{align*}
and
\begin{align*}
\cL\in\cco\cdot\cocap\cdot\sC
&\Longleftrightarrow\Sigma^*\setminus\cL\in\cocap\cdot\sC \\
&\Longleftrightarrow\Sigma^*\setminus\cL\in\sC
\AND\Sigma^*\setminus\cL\in\cco\cdot\sC \\
&\Longleftrightarrow\cL\in\cco\cdot\sC\AND\cL\in\cco\cdot\cco\cdot\sC \\
&\Longleftrightarrow\cL\in\cco\cdot\sC\AND\cL\in\sC \\
&\Longleftrightarrow\cL\in\cocap\cdot\sC.
\end{align*}
\end{proof}

For many operators, it is the case that $\sC\subseteq\op\cdot\sC$ for every 
$\sC$, because the definitions of these classes include an additional 
certificate or advice string that can be ignored.

\begin{proposition}
$\id\subseteq\op$, where $\op=\cpoly,\oplus,\cBP,\cP,\cN\text{ or }\exppad$.
\end{proposition}

\begin{proof}
Fix $\cL\in\sC$. Then $\cL\in\op\cdot\sC$ for each possible choice of $\op$:
\begin{itemize}
\item If $\op=\cpoly$, take $\cL^\prime=\cL$ and $p(n)=\epsilon$ for all $n\in\N$
  in the definition of $\cpoly\cdot\sC$.
\item If $\op=\oplus,\cBP,\cP,\text{ or }\cN$, take $\cL^\prime=\cL$ and
  $p(n)=0$ for all $n\in\N$ in the definition of $\op\cdot\sC$.
\item If $\op=\exppad$, take $\cL^\prime=\cL$ and $f(n)=\epsilon$ for all
  $n\in\N$ in the definition of $\exppad\cdot\sC$.
\end{itemize}
\end{proof}
Since the condition $\cL\in\cBP\cdot\sC$ is a strengthening of the condition that
$\cL\in\cP\cdot\sC$, the following is immediate.
\begin{proposition}
$\cBP\leq\cP$.
\end{proposition}

We next consider some commutativity properties.

\begin{proposition}
$\cco\cdot\op=\op\cdot\cco$, where $\op=\cBP,\cP,\text{ or }\cpoly$.
\end{proposition}

\begin{proof}
For each of the possible choices of $\op$, the definition of $\op\cdot\sC$ has
the following form:
\[
\op\cdot\sC:=\{\cL\subseteq\Sigma^*:(\exists\cL^\prime\in\sC)
\psi(\cL,\cL^\prime)\},
\]
where $\psi(\cL,\cL^\prime)$ is a proposition having the property that
\[
\psi(\cL,\Sigma^*\setminus\cL^\prime)\Longleftrightarrow
\psi(\Sigma^*\setminus\cL,\cL^\prime).
\]
Thus,
\begin{align*}
\cco\cdot\op\cdot\sC
&=\{\cL\subseteq\Sigma^*:(\exists\cL^\prime\in\sC)\psi(\Sigma^*\setminus\cL,\cL^\prime)\} \\
&=\{\cL\subseteq\Sigma^*:(\exists\cL^\prime\in\sC)
\psi(\cL,\Sigma^*\setminus\cL^\prime)\} \\
&=\{\cL\subseteq\Sigma^*:(\exists\cL^\prime\in\cco\cdot\sC)
\psi(\cL,\cL^\prime)\} \\
&=\op\cdot\cco\cdot\sC
\end{align*}
for each possible choice of $\op$.
\end{proof}
A similar argument, based on the structure of the definitions of the relevant
operators, can be used to show that $\cpoly$ commutes with $\oplus$, $\cN$, and
$\cP$.

\begin{proposition}
If complexity classes are assumed to be closed under polynomial-time reductions, then
$\cpoly\cdot\op=\op\cdot\cpoly$, where $\op=\oplus,\cN,\text{ or }\cP$.
\end{proposition}

\begin{proof}
We say that $\cL\in\cpoly\cdot\op\cdot\sC$ if there exist $\cL^\prime\in\sC$,
$p(n)=\Oh(n^*)$, and $|q(n)|=\Oh(n^*)$ such that for every $x\in\Sigma^*$,
\[
x\in\cL\Longleftrightarrow
(Qy\in\Sigma^{p(|\inner{x,q(|x|)}|)})[\inner{\inner{x,q(|x|)},y}\in\cL^\prime],
\]
where $Q$ is the quantifier in the definition of the operator that is being
considered. Similarly, we say that $\cL\in\op\cdot\cpoly\cdot\sC$ if there exist
$\cL^\prime\in\sC$, $p(n)=\Oh(n^*)$, and $|q(n)|=\Oh(n^*)$ such that for every
$x\in\Sigma^*$,
\[
x\in\cL\Longleftrightarrow
(Qy\in\Sigma^{p(|x|)})[\inner{\inner{x,y},q(|\inner{x,y}|)}\in\cL^\prime].
\]
The condition $\cL\in\cpoly\cdot\op\cdot\sC$ is equivalent to the condition that
there are $\cL^\prime\in\sC$, $\bar p\in\Oh(n^*)$, and $q\in|\Oh(n^*)|$ such
that for all $x\in\Sigma^*$,
\[
x\in\cL\Longleftrightarrow
(Qy\in\Sigma^{\bar p(|x|)})[\inner{\inner{x,q(|x|)},y}\in\cL^\prime].
\]
For instance, if $\cL\in\cpoly\cdot\op\cdot\sC$, then we can set $\bar
p(n)=p(N)$, where $N=|\inner{x,q(|x|)}|$ for $|x|=n$. Likewise, in the
conditions for $\cL\in\op\cdot\cpoly\cdot\sC$ we can replace $q$ with a $\bar q$
so that
\[
x\in\cL\Longleftrightarrow
(Qy\in\Sigma^{p(|x|)})[\inner{\inner{x,y},\bar q(|x|)}\in\cL^\prime].
\]
The rewritten conditions for $\cL\in\cpoly\cdot\op\cdot\sC$ and
$\cL\in\op\cdot\cpoly\cdot\sC$ are then equivalent to each other because a
mapping between $\inner{\inner{x,z},y}$ and $\inner{\inner{x,y},z}$ is
polynomial-time computable.
\end{proof}

\begin{proposition}
If complexity classes are assumed to be closed under polynomial-time reductions, then
$\cco\cdot\oplus=\oplus\cdot\cco$.
\end{proposition}

Finally, we consider the properties of the operator $\sC\mapsto\sP^\sC$.
\begin{proposition}
For any complexity class $\sC$,
\begin{enumerate}
\item $\sC\subseteq\sP^\sC$;
\item $\cco\cdot\sC\subseteq\sP^\sC$;
\item $\cco\cdot\sP^\sC=\sP^{\cco\cdot\sC}=\sP^\sC$.
\end{enumerate}
\end{proposition}
\begin{proof}
If $\cL\in\sC$ and $f$ is the decision function for $\cL$, then
$\cL\in\sP^f\subseteq\sP^\sC$. Hence $\sC\subseteq\sP^\sC$. Moreover,
$\sP^f$ is a symmetric class for every $f$, and so
\begin{align*}
\cL\in\cco\cdot\sP^\sC
&\Longleftrightarrow\Sigma^*\setminus\cL\in\sP^\sC \\
&\Longleftrightarrow(\exists f\in\sC)[\Sigma^*\setminus\cL\in\sP^f] \\
&\Longleftrightarrow(\exists f\in\sC)[\cL\in\sP^f] \\
&\Longleftrightarrow\cL\in\sP^\sC,
\end{align*}
and
\begin{align*}
\cL\in\sP^{\cco\cdot\sC}
&\Longleftrightarrow(\exists f\in\sC)[\cL\in\sP^{1-f}] \\
&\Longleftrightarrow(\exists f\in\sC)[\cL\in\sP^f] \\
&\Longleftrightarrow\cL\in\sP^\sC,
\end{align*}
because $\sP^f=\sP^{1-f}$ for every oracle $f$. Thus, (1) and (3) are true. (2)
then follows immediately, because
$\sC\subseteq\sP^\sC\Longrightarrow\cco\cdot\sC
\subseteq\cco\cdot\sP^\sC\Longrightarrow\cco\cdot\sC\subseteq\sP^\sC$.
\end{proof}

\begin{proposition}
For any non-trivial class $\sC$ that is closed under joins and polynomial-time 
reductions, $\cpoly\cdot\sP^\sC=\sP^{\cpoly\cdot\sC}$. If $\sC$ is also closed under
polynomial majorities, then $\cBP\cdot\sP^\sC=\sP^{\cBP\cdot\sC}$.
\end{proposition}

\begin{proof}[Proof of first equation]
First, we show that it is unconditionally the case that
$\sP^{\cpoly\cdot\sC}\subseteq\cpoly\cdot\sP^\sC$. Suppose that
$\cL\in\sP^{\cpoly\cdot\sC}$. Then there is a polynomial-time algorithm with
$f$-oracle that computes $\cL$, where $f$ is a decision function for a language
$\cL^\prime\in\cpoly\cdot\sC$. By the definition of the $\cpoly$ operator, there
exists a language $\cL^{\prime\prime}\in\sC$ and an advice function
$|p(n)|=\Oh(n^*)$ such that $x\in\cL^\prime$ if and only if
$\inner{x,p(|x|)}\in\cL^{\prime\prime}$.

Let $g$ indicate the decision function for $\cL^{\prime\prime}$. Also,
let $q(n)=\Oh(n^*)$ denote the time bound for the $\sP^f$ algorithm for $\cL$, so
that the question of whether $x\in\cL$ is decided in at most $q(|x|)$
computational steps. Define $P:\N\rightarrow\Sigma^*$ so that, for each
$n\in\N$, $P(n)$ is the concatenation $p(0)p(1)\ldots p(q(n))$.

The following algorithm in $\cpoly\cdot\sP^{g}$ decides whether $x\in\cL$:
\begin{enumerate}
\item The advice function is $P$. Note that $|P(n)|=\Oh(n^*)$, because for
  sufficiently large $n$ $|P(n)|$ is at most $q(n)|p(q(n))|$.
\item Follow the $\sP^f$ algorithm for $\cL$ exactly, except when there is an
  oracle call.
\item When an oracle call to $f$ occurs with the string $y\in\Sigma^*$, replace
  it with an oracle call to $f^\prime$ with the string $\inner{y,p(|y|)}$. This
  oracle call is possible because $P(|x|)$ contains the advice strings for all
  $y$ that are short enough for the $\sP^f$ algorithm to be able to query the
  oracle.
\end{enumerate}
Thus, we have $\cL\in\cpoly\cdot\sP^{g}\subseteq\cpoly\cdot\sP^\sC$, and we
can conclude that $\sP^{\cpoly\cdot\sC}\subseteq\cpoly\cdot\sP^\sC$ in all cases.

For the inclusion $\cpoly\cdot\sP^\sC\subseteq\sP^{\cpoly\cdot\sC}$, suppose that
$\cL\in\cpoly\cdot\sP^\sC$. This means that there exists a $\sP^f$ algorithm that
decides $\cL$ when provided with some advice function $|p(n)|=\Oh(n^*)$, where
$f$ is the decision function of some $\cL^\prime\in\sC$. 
Define $g:\Sigma^*\ra\Sigma$ according to the following rules:
\begin{itemize}
\item If $x=0\inner{y,z}$, then $g(x)$ is equal to the $z$th bit of
  $p(|y|)$.
\item If $x=1y$, then $g(x)=A(y)$.
\end{itemize}
The language $\cL^{\prime\prime}$ determined by $g$ is the join of two
languages. One, which we will call $\cL_1^{\prime\prime}$, is the set of all
$\inner{y,z}$ such that the $z$th bit of $p(|y|)$ is 1; the second language is
$\cL^\prime$.

We claim that $\cL^{\prime\prime}$ lies in $\cpoly\cdot\sC$. To prove this, it is
enough to show that $\cL_1^{\prime\prime}\in\sP/\poly$. Then
$\sP/\poly\subseteq\cpoly\cdot\sC$ (we know that $\sP\subseteq\sC$ because $\sC$
is assumed to be nontrivial and closed under polynomial-time reductions), and
$\cL^\prime\in\sC\subseteq\cpoly\cdot\sC$, so
$\cL^{\prime\prime}\in\cpoly\cdot\sC$ by the hypothesis that $\sC$, and therefore
$\cpoly\cdot\sC$, is closed under joins.

To see that $\cL_1^{\prime\prime}\in\sP/\poly$, let $|P(n)|=\Oh(n^*)$ be the
function defined by the concatenation 
\[
P(n)=p(0)p(1)\ldots p(n).
\] 
Then, given $\inner{y,z}$, $P$ can be used as an advice function to check 
whether the $z$th bit of $p(|y|)$ is 1.

Hence $\cL^{\prime\prime}\in\cpoly\cdot\sC$. To show that 
$\cL\in\sP^{\cpoly\cdot\sC}$, we show that $\cL\in P^{g}$. The following 
is a $\sP^{g}$ algorithm for deciding whether $x\in\cL$:
\begin{enumerate}
\item First, extract the advice string $p(|x|)$ from the $g$-oracle. Make
  the oracle queries $0\inner{x,j}$, $j\leq|p(|x|)|$ until the entirety of
  $p(|x|)$ has been recorded.
\item Carry out the rest of the computation according to the $\cpoly\cdot\sP^f$
  algorithm. Replace oracle queries to $f$ about the string $y$ with oracle
  queries to $g$ about the string $y$.
\end{enumerate}
Therefore $\cL\in\sP^{\cpoly\cdot\sC}$, concluding the proof that 
$\sP^{\cpoly\cdot\sC}=\cpoly\cdot\sP^\sC$.
\end{proof}
