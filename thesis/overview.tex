\section{Models of Computation}

\subsection{Complexity Classes and Decision Problems}

A computational problem is a question about a particular object of input, such
as ``Is the input $x$ a prime number?'' or ``What is the greatest common divisor
of the integers $x$ and $y$?'' It is assumed that the input to the question, as
well as the answer, can be encoded as a finite string of zeros and ones---i.e.,
as an element of $\Sigma^*$. (We denote by $\Sigma$ the set $\{0,1\}$ and by
$S^*$ the set of all finite strings whose characters consist of elements of the
set $S$.) Thus, a computational problem can be modeled as a function
$f:\Sigma^*\ra\Sigma^*$.

In general, a complexity class is a set of computational problems
$f:\Sigma^*\ra\Sigma^*$, usually interpreted as the set of all problems that are
tractable within a specified computational model. For this project, we are
interested in \textit{decision problems}, which are computational problems whose
answer is either yes or no (encoded as 1 and 0, respectively). Within the
functional framework, decision problems are formalized as functions
$d:\Sigma^*\ra\Sigma$. Such functions can be identified with the set
$\{x\in\Sigma^*:d(x)=1\}$. For this reason, we will identify decision problems
with \textit{languages}, or subsets $\cL$ of $\Sigma^*$. We will also conflate 
a language with its corresponding decision function when it is convenient to do
so.

\subsection{Notation and Conventions}

At this point, it will be useful to identify some conventions. The set $\N$ of
natural numbers is assumed to contain zero. We use big-O notation to describe
the size of functions: for a pair of functions $f:\N\ra\N$ and $g:\N\ra\N$, we
write $f=\Oh(g)$ (or sometimes $f(n)=\Oh(g(n))$) if there exists a real constant
$C$ such that $f(n)\leq Cg(n)$ for every $n\in\N$. We also write $f(n)=\Oh(n^*)$
to indicate that that there exists $k\in\N$ such that $f(n)=\Oh(n^k)$. Exponents
are used to write strings having the same character repeated multiple times, so
that $0^4=0000$, for example.

\subsection{Classical Computation} \label{classical-models}

The most commonly used model of computation in classical complexity theory is
the \textit{Turing machine}. A (deterministic) \textit{Turing machine} with
$k\geq 2$ tapes, abbreviated TM, is a triple $M=(\Gamma, Q, \delta)$ containing 
the following data:
\begin{enumerate}[(i)]
\item A set $\Gamma$ of \textit{symbols}, called the \textit{alphabet}, which
  must include the blank symbol $\square$, the start symbol $\triangleright$,
  and the numerals 0 and 1;
\item A set $Q$ of \textit{states} of $M$, one of which is the starting state
  $q_\start$, and another of which is the halting state $q_\halt$;
\item A \textit{transition function} $\delta:Q\times\Gamma^k\ra
  Q\times\Gamma^{k-1}\times\{\Left,\Stay,\Right\}^k$ satisfying
  \[
  \delta(q_\halt,s_1,\ldots,s_k)= (q_\halt,s_2,\ldots,s_k,\Stay,\ldots,\Stay)
  \]
  for every $s_i\in\Gamma$.
\end{enumerate}

A \textit{configuration} for a Turing machine $M=(\Gamma, Q, \delta)$ with $k$
tapes is a tuple $c=(q,i_1,\ldots,i_k,t)$, where $q\in Q$, $i_1,\ldots,i_k$ are
natural numbers, and $t:\{1,\ldots,k\}\times\N\ra\Gamma$ is a \textit{tape
function} such that $t(j,\ell)=\square$ for all sufficiently large $\ell$.

An \textit{initial configuration} is a configuration $c=(q_\start,0,\ldots,0,t)$
satisfying the following conditions:
\begin{enumerate}[(i)]
\item $t(j,0)=\triangleright$ for $j\in\{1,\ldots,k\}$;
\item $t(j,\ell)=\square$ for $j>1$ and $\ell\in\N$;
\item There exists a natural number $N\geq 1$ such that $t(1,\ell)\neq\square$
  for all $\ell<N$ and $t(1,\ell)=\square$ for all $\ell\geq N$.
\end{enumerate}
The string $t(1,1)\ldots t(1,N-1)$ is the \textit{input}. If $N=1$, the input is
considered to be the empty string $\epsilon$.

A \textit{halted configuration} is a configuration with $q=q_\halt$. If there
exists a natural number $N$ such that $t(k,\ell)\neq\square$ for $\ell\leq N$
and $t(k,\ell)=\square$ for $\ell>N$, then the string $t(k,0)\ldots t(k,N)$ is
the \textit{output}; otherwise, $\epsilon$ is the output.

Let $c=(q,i_1,\ldots,i_k,t)$ and
$c^\prime=(q^\prime,i_1^\prime,\ldots,i_k^\prime,t^\prime)$ be configurations
for $M=(\Gamma, Q,\delta)$. The configuration $c^\prime$ \textit{succeeds} $c$
($c\ra c^\prime$) if
\[
\delta(q,t(1,i_1),\ldots,t(k,i_k))=
(q^\prime,t^\prime(2,i_2),\ldots,t^\prime(k,i_k),
\mathsf{X}_1,\ldots,\mathsf{X}_k),
\]
where $t(j,\ell)=t^\prime(j,\ell)$ if $j=1$ or $\ell\not\in\{i_2,\ldots,i_k\}$,
and
\begin{align*}
i_j^\prime &=\max\{0,i_j-1\}\text{ if }\mathsf{X}_j=\Left, \\
i_j^\prime &=i_j\text{ if }\mathsf{X}_j=\Stay, \\
i_j^\prime &=i_j+1\text{ if }\mathsf{X}_j=\Right.
\end{align*}
A \textit{computation} for a machine $M$ is a sequence $c_0\ra c_1\ra\ldots\ra 
c_n$, where $c_0$ is an initial configuration and $c_n$ is a halted 
configuration. The input of $c_0$ and the output of $c_n$ are the \textit{input}
and \textit{output} of the computation, respectively. The machine $M$ 
\textit{computes} the function $f:\Sigma^*\ra\Sigma$ if the computation for $M$ 
with input $x$ has output $f(x)$. We also write $M(x)$ to indicate the output of
$M$ with input $x$. If $M(x)=1$, the machine \textit{accepts} the input; if
$M(x)=0$, the machine \textit{rejects} the input.

The \textit{computation time} $T_M(x)$ of the machine $M$ with input $x$ is the 
length of the computation of $M$ with input $x$. For a function $f:\N\ra\N$, $M$
\textit{runs in $f(n)$-time} if there exists a constant $C$ such that 
$T_M(x)\leq C\cdot f(|x|)$ for every $x\in\Sigma^*$. $M$ runs in 
\textit{polynomial time} if it runs in $f(n)$-time for some $f(n)=\Oh(n^*)$. The 
function class of all $f:\Sigma^*\ra\Sigma^*$ that are polynomial-time 
computable is given by $\msf{FP}$.

There are several variations of the Turing machine concept. In a 
\textit{non-deterministic} Turing machine, there are two transition functions, 
and a computation is a tree of configurations rather than a sequence. A 
probabilistic Turing machine also has two transition functions, but at each 
stage of the computation the machine tosses a coin to decide which transition 
function it uses. These variations are often useful, but it is usually enough to
use a standard, deterministic Turing machine along with an additional string 
$y\in\Sigma^*$ placed alongside the input. $y$ can then represent either a 
sequence of random bits/coin tosses or a particular branch of a 
non-deterministic computation.

All Turing machines are assumed to have an alphabet consisting entirely of
$\Sigma=\{0,1\}$. Any additional symbols are assumed to be encoded in some
way. For example, if the input is a tuple, such as $\inner{x,y}$ for
$x,y\in\Sigma^*$, then we could encode the symbol 0 as 00, 1 as 01, and the
separating comma as 10. We will also occasionally want to consider a natural
number as being the input or output of a computational process. We therefore fix
a bijection $e:\N\ra\Sigma^*$ allowing us to identify natural numbers with the
strings in $\Sigma^*$. For definiteness, we can take $e(n)$ to be the result of
removing the initial 1 from the binary representation of $n+1$.

Another classical model of computation is the \textit{Boolean circuit}. A
Boolean circuit is an acyclic graph with one or more sources (vertices with no
incoming edges) and exactly one sink (a vertex with no outgoing edges). Each
vertex that is not a source is labeled with $\wedge$, $\vee$, or
$\neg$ and represent the \textit{gates} of the circuit. Vertices labeled with 
$\neg$ must have a fanin---i.e., number of
incoming edges---of 1. Each source is labeled with a unique element of
$\{1,\ldots,n\}$, where $n$ is the number of sources.

For a Boolean circuit $C$, a \textit{value function} $v:V\ra\Sigma$ is a
function on the set $V$ of vertices of $C$ satisfying the following properties:
\begin{itemize}
\item If the vertex $x$ is labeled with $\wedge$, then $v(x)=1$ if and only if
  $v(y)=1$ for every predecessor $y$ of $x$.
\item If the vertex $x$ is labeled with $\vee$, then $v(x)=1$ if and only if
  $v(y)=1$ for some predecessor $y$ of $x$.
\item If the vertex $x$ is labeled with $\neg$, then $v(x)=1$ if and only if
  $v(y)=0$, where $y$ is the unique predecessor of $x$.
\end{itemize}
For each assignment of 0 or 1 to a source vertex, there is a unique value
function that matches the assignment. (Since $C$ is finite and acyclic, if this
were not the case there would be a minimal vertex for which a unique value of
$v$ is not determined, but such a vertex is either a source or determined by its
predecessors by minimality.)

Thus, given a Boolean circuit $C$, we can define a function $\bar
C:\Sigma^n\ra\Sigma$ according to the following procedure for $x\in\Sigma^n$,
let $v:V\ra\Sigma$ be the value function that assigns the $k$th bit of $x$ to
the source with the label $k$; then, set $\bar C(x)=v(y)$, where $y$ is the sink
of $C$. In this way, the circuit $C$ can be thought of as computing the function
$\bar C$.

Unlike Turing machines, Boolean circuits require the input to be of fixed 
length. A natural solution to this problem is to consider a family $\{C_n:n\geq 
1\}$ of circuits, which is then considered to compute  the function $\bar 
C:\Sigma^*\ra\Sigma$ defined by $\bar C(x)=\bar C_n(x)$ for all $x\in\Sigma^*$, 
where $n$ is the length of $x$. However, this model is actually more powerful 
than a Turing machine, because it can compute any unary language (languages 
consisting entirely of strings of 1s). When the size of $C_n$ is limited to a 
polynomial of $n$, the resulting complexity class is the uncountable class 
$\sP/\poly$ rather than the usual polynomial-time computable class $\sP$. 
Intuitively, since there is a separate circuit for each possible input length 
$n$, it is possible to encode an amount of advice into each circuit that is a 
polynomial of $n$ in length.

\subsection{Quantum Computation}\label{quantum-model}

It remains to define a computational model for quantum computation. Rather than
using classical bits, which must exist in a state of 0 or 1, quantum computation
uses quantum bits, or \textit{qubits}. A qubit is a normalized element
$a\ket{0}+b\ket{1}$ of $\C^2$, where $a,b\in\C$ and $\ket{0}$ and $\ket{1}$ are
orthonormal with respect to the usual inner product on $\C^2$ (so that
$|a|^2+|b|^2=1$). There are several possible models by which the qubits can be
employed for computation; for definiteness, the quantum circuit will be our
standard choice. Just as the 0-1 values are changed as they move through a
classical circuit, the qubits of a quantum circuit are transformed unitarily as
they pass through the vertices of a circuit.

\begin{definition}
A \textit{quantum circuit} is a finite acyclic graph satisfying the following
properties:
\begin{itemize}
\item There are $n$ sources, each of which is labeled with an element of the set
  $\{1,\ldots,n\}$.
\item Each source has a fanout of 1, and each sink has a fanin of 1.
\item If a vertex is neither a sink nor a source, then its fanin must equal its
  fanout, which can be at most 3.
\item Each vertex $x$ that is neither a sink nor a source is labeled with a
  $2^{k_x}$-dimensional unitary transformation, where $k_x$ is the fanin of
  $x$. The vertex is also labeled with a bijective function $f_x:E_i^x\ra
  E_o^x$, where $E_i^x$ is the set of incoming edges for $x$ and $E_o^x$ is the
  set of outgoing edges for $x$.
\end{itemize}
\end{definition}
Note that, from each source, there is a canonical path to follow:
\begin{itemize}
\item From the source, follow the only edge forward.
\item When arriving at the vertex $x$ from the edge $e$, follow the edge
  $f_x(e)$ forward.
\item The path ends when a sink is reached.
\end{itemize}
Each such path terminates at a different sink, and every sink is the terminal
vertex of some such path. Thus, there is a natural bijection between sources and
sinks, and we can assign the corresponding element of $\{1,\ldots,n\}$  to
each of the sinks.

As with classical circuits, the input to a quantum circuit is a string of
bits---in this case, a direct product of qubits of the form $\ket{j}$,
$j=0,1$ (we denote
$\ket{j_1}\otimes\ldots\otimes\ket{j_n}=\ket{j_1}\ldots\ket{j_n}=\ket{j_1\ldots 
j_n}$. The qubits then travel along the circuit according to the canonical
paths, being transformed by the unitary maps at each vertex. Once each qubit
reaches its sink, the initial bit string will have been transformed into a new
state $\sum\alpha_{j_1j_2\ldots j_n}\ket{j_1j_2\ldots j_n}$. For a circuit $C$
and initial state $\varphi$, we denote the final state by $\bar
C(\ket{\varphi})$.

To be explicit, this is the procedure by which the initial state $\ket{\varphi}$
is transformed into $\bar C(\ket{\varphi})$:
\begin{enumerate}
\item If necessary, add vertices so that all canonical paths have the same
  length and each vertex occurs at the same length along any canonical path
  passing through it. We do this by introducing new vertices with fanin 1 and
  labeling new non-terminal vertices with the identity transformation. Denote
  the new length of each canonical path by $N$.
\item For $1\leq k\leq N$, define a unitary transformation $U_k$:
  \begin{enumerate}
  \item Reorder the set $\{1,\ldots,n\}$ so that canonical paths passing
    through the same $k$th vertex are assigned adjacent numbers. Denote the
    resulting permutation of qubits by $W$.
  \item Number the transformations labeling each $k$th vertex so that they
    appear in the same order indicated by the permutation $W$:
    $V_1,V_2,\ldots,V_{r_k}$.
  \item Set $U_k=W^*(V_1\otimes V_2\otimes\ldots\otimes V_{r_k})W$.
  \end{enumerate}
\item Set $\bar C(\ket{\varphi})=U_N\ldots U_2U_1\ket{\varphi}$.
\end{enumerate}
Finally, we obtain the result of the computation from $\bar C(\ket{\varphi})$ by
\textit{measuring} it. If $\bar C(\ket{\varphi})=\Sigma_x\alpha_x\ket{x}$, where
the sum is over all elements of $\Sigma^n$, then the result of the measurements
is $\ket{x}$ with probability $|\alpha_x|^2$.

To make meaningful discussion of quantum complexity theory possible, it is 
necessary to restrict the permitted operations to a finite set of 
\textit{universal operations}. This means that any unitary matrix (of dimension 
$\geq 3$) can be approximated to arbitrary precision by the finite collection of
operators, in the same way that $\wedge$, $\vee$, and $\neg$ gates suffice for 
the purposes of classical computation. Moreover, this approximation can be 
accomplished efficiently:
\begin{theorem}[Solovay-Kitaev, \cite{kitaev1997quantum}]
There exists a finite set $F$ of unitary operators, each having dimension $\leq 
3$, such that $F$ is an \textit{efficient universal gate set}, in the following 
sense:

Let $d\geq 3$ be an integer, and let $\epsilon>0$. There exists a positive 
integer $\ell\leq 100(d\log 1/\epsilon)^3$ such that for every $d\times d$ 
unitary matrix $U=(U_{jk})$, there exist unitary matrices $U_1,\ldots,U_\ell$ 
such that for each $j,k\in\{1,\ldots,d\}$,
\[
|U_{jk}-(U_\ell\ldots U_1)_{jk}|<\epsilon,
\]
where each $U_j$ corresponds to applying an operation from $F$ to at most 3 of 
$d$ qubits.
\end{theorem}
One possible choice of universal operators is the Hadamard gate
$H=\frac{1}{\sqrt{2}}\left(\begin{smallmatrix}
1 & 1 \\ 1 & -1\end{smallmatrix}\right)$,
the Toffoli gate 
$I_6\oplus\left(\begin{smallmatrix}
0 & 1 \\ 1 & 0\end{smallmatrix}\right)$
(where $I_6$ denotes the $6\times 6$ identity matrix), and the phase shift gate
$\left(\begin{smallmatrix}
1 & 0 \\ 0 & i\end{smallmatrix}\right)$.

\begin{definition}
Let $\{C_n:n\geq 1\}$ be a family of quantum circuits such that $C_n$ has
$m_n\geq n$ sources. $\{C_n:n\geq 1\}$ \textit{computes $f:\Sigma^*\ra\Sigma$
with probability $p$} if for every $x\in\Sigma^*$, when $\bar
C_n(\ket{x0^{m_n-n}})$ is measured, the first bit is equal to $f(x)$ with
probability $\geq p$.
\end{definition}

As in the classical case, to define the class of problems that are quantum
computable in polynomial time, it is not only necessary to restrict the size of
$C_n$ to a polynomial of $n$, but also to require that it is possible to
classically compute a description of $C_n$ from an input of $n$ in
polynomial time. Without this restriction, we would admit as computable many
languages that are not generally thought of as computable. This restriction of
circuits also works in the classical case, reducing the resulting complexity
class from $\sP/\poly$ to $\sP$.

Many complexity classes of interest are derived from more traditional complexity
classes by replacing the classical method of computation with the quantum
one. Doing so to the class $\sP$ results in the class $\msf{BQP}$, doing so to
the class $\msf{MA}$ results in the class $\msf{QMA}$, and so on. Using this
observation as a basis allows us to imagine a non-rigorous ``pseudo-operator''
$\sC\mapsto\mathpzc{Q}\cdot\sC$ that transforms a classical complexity class $\sC$
into its quantum counterpart $\mathpzc{Q}\cdot\sC$. Since operators must be able to
act on complexity classes independently of the underlying means of computation,
we cannot make $\mathpzc{Q}$ into a rigorously defined operator. Nevertheless, it is
intuitively useful to think of $\mathpzc{Q}$ as an operator and note one property in
particular: $\sC\subseteq\mathpzc{Q}\cdot\sC$ for every $\sC$. In other words, any
computations that are possible classically must also be possible in the quantum 
world. This is not immediately obvious from our models of computation, because, 
for example, classical circuits allow for non-reversible operations (e.g. one 
cannot determine from the value of an $\wedge$-vertex the value of its 
predecessors) while quantum operations are unitary and therefore reversible by 
necessity. However, any classical operations have reversible quantum versions, 
generally implemented by means of additional ``scratch qubits'' that are assumed
to be zero at the outset. For instance to implement the $\wedge$-operator, we 
apply the unitary mapping
\[
\ket{xy}\ket{z}\mapsto\ket{xy}\ket{z+_2xy},
\]
where $+_2$ indicates addition modulo 2. As long as the scratch qubit
$\ket{z}$ is initialized to 0, the bit $\ket{z+_2xy}$ will be equal to the
classical $\wedge$-operator on $x$ and $y$. The need for extra scratch bits is
the reason we allow trailing zeros in the input to a quantum circuit.

\subsection{Universal Turing Machines}

Turing machines should likewise be encoded as elements of $\Sigma^*$. We fix a
surjective mapping $\alpha\mapsto M_\alpha$ from $\Sigma^*$ to the set of Turing
machines. For convenience, this mapping should have the property that for every
Turing machine $M$ there exist infinitely many $\alpha$ such that
$M=M_\alpha$. The encoding is chosen so that the following theorem is true:
\begin{theorem}\label{universal-machine}
There exists a Turing machine $U$ such that $U(x,\alpha)=M_\alpha(x)$ for all
$x,\alpha\in\Sigma^*$ with the property that if $T(x,\alpha)$ is the computation
time for $M_\alpha(x)$, then the computation time for $U(x,\alpha)$ is
$CT(x,\alpha)\log T(x,\alpha)$, where $C$ depends only on the number of tapes
and states of $M_\alpha$.
\end{theorem}
Functions are generally restricted to those that are
\textit{time-constructible}, meaning that the function $T:\N\ra\N$ can be
computed by a Turing machine in time $\Oh(T(n))$ and $T(n)\geq n$ for all
$n\in\N$.

\section{Operators and Relativization}

Each class in Complexity Zoology's data set can be
\textit{relativized}. Informally, relativization is the process of taking a
particular computational problem $f:\Sigma^*\ra\Sigma^*$ as a black box that
can be given any input, and the answer is instantaneously given. The black box
is called the \textit{oracle} and the process of giving an input to the black
box is referred to as \textit{querying} the oracle. For a complexity class
$\sC$, we denote by $\sC^f$ the complexity class with the same computational
model as $\sC$, but for which queries to an oracle $f$ are allowed. At a
minimum, this means that if $f:\Sigma^*\ra\Sigma$ is a decision function, then
the associated language $\cL$ lies in $\sC^f$. Moreover, we expect that
$\sC\subseteq\sC^f$, since the oracle can simply be ignored during any
computation. For a pair of complexity classes $\sC_1,\sC_2$, we also set 
$\sC_1^{\sC_2}=\bigcup_{f\in\sC_2}\sC_1^f$.

Each of the computational models we consider is powerful enough to encode finite
changes to an oracle. For this reason, if $f$ and $g$ are oracles that 
differ only on a finite subset of $\Sigma^*$, then $\sC^f=\sC^{g}$. This 
fact is crucial to the concept of a \textit{random oracle}.
\begin{lemma}[\cite{fortnow2018zero}]
Let $r:\Sigma^*\ra\Sigma$ be chosen uniformly at random, so that $\Pr[r(x)=1]=1/2$ 
for each $x\in\Sigma^*$. If $\mathcal{F}$ is any set of functions 
$\Sigma^*\ra\Sigma$ that is closed under finite differences between functions, then
$\Pr[r\in\mathcal{F}]=0\text{ or }1$.
\end{lemma}
\begin{proof}
The \textit{Kolmogorov zero-one law} states that if $S=\{X_j:j\in\N\}$ is a 
countable set of independent random variables and $E$ is an event that is 
independent of each finite subset of $S$, then $\Pr[E]=0\text{ or }1$. For our 
purposes, $X_j=r(j)$ and $E$ is the event $r\in\mathcal{F}$. This event is 
independent of finitely many choices of values $r(j)$, because the closure property
of $\mathcal{F}$ implies that changing finitely many $r(j)$ does not affect whether
$r\in\mathcal{F}$. Thus, the zero-one law proves the lemma.
\end{proof}
By the previous remark, the set
\[
\mathcal{F}=\{f:f\text{ is a function }\Sigma^*\ra\Sigma\text{ such that }
\sC_1^f\subseteq\sC_1^f\}
\]
for a pair of complexity classes $\sC_1$ and $\sC_2$ is closed under finite 
differences. We therefore have the following result:
\begin{theorem}
If $r:\Sigma^*\ra\Sigma$ is chosen uniformly at random, then 
$\sC_1^r\subseteq\sC_2^r$ with probability $0$ or $1$.
\end{theorem}
As a consequence of this theorem, we can consider whether $\sC_1^r\subseteq\sC_2^r$
with respect to ``the'' random oracle $r$.

Contrary to what the notation would suggest, $\sC\mapsto\sC^f$ is not a map on 
the set of complexity classes. For example, there is an oracle $f$ such that 
$\sP^f\neq\NP^f$, but if $\sC\mapsto\sC^f$ were a function on the set of 
complexity classes, this would imply $\sP\neq\NP$, which is an open problem. 
Instead, the action $\sC\mapsto\sC^f$ transforms the computational model 
itself. In the case of complexity classes involving Turing machines, the Turing
machines are replaced with \textit{oracle Turing machines} capable of querying 
an oracle.

Unfortunately, there is no uniform way to define $\sC^f$ from $\sC$ that works
for every complexity class in the data set. Thus, we formally consider a
complexity class to be a family of sets $\sC=\{\sC^f:f\text{ is a function 
}\Sigma^*\ra\Sigma^*\}$. We often identify $\sC$ with $\sC^t$, where $t$ is the 
trivial oracle defined by $t(x)=0$ for all $x\in\Sigma^*$, and the definition of
$\sC^f$ will depend on the computational model for the class in question. For 
classes defined in terms of Turing machines, we use \textit{oracle} Turing 
machines, which have a special oracle tape which can be used to query the oracle 
and receive a response. In circuit models of computation, special gates are used 
to query the oracle.

Oracle relativization has been useful in assessing the viability of several common 
proof techniques for answering complexity theoretic questions. For example, the 
celebrated theorem of Baker, Gill, and Solovay states that there exist oracles $f$ 
and $g$ such that $\sP^f=\NP^f$ and $\sP^g\neq\NP^g$ 
\cite{baker1975relativizations}. (For definitions of the classes $\sP$ and $\NP$, 
see Subsection \ref{class-definitions}.) Therefore, if a proof technique 
relativizes---that is, if the proof is independent of any oracles that are applied 
to the complexity classes involved---it cannot be used to settle the $\sP$ vs. 
$\NP$ question. This obstacle is known as the \textit{relativization barrier}.

Later, Aaronson and Wigdersons introduced a refinement known as the 
\textit{algebrization barrier} \cite{aaronson2009algebrization}. More recently, 
Aydinlio\u{g}lu and Bach reformulated the idea of an algebrizing proof into one 
that relativizes with respect to a class of oracles they refer to as 
\textit{affine}. In this thesis, we refer to these oracles as \textit{algebraic}.