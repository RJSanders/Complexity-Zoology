In this chapter, we describe the algorithm that Complexity Zoology follows. Here is a high-level description of the procedure:
\begin{enumerate}
    \item \textbf{Read input:} The program parses the plain-text input files (one  
      for classes and one for operators).
    \item \textbf{Process equalities:} The input file contains statements of the 
      form $\mathtt{C1=C2}$, where $\mathtt{C1}$ and $\mathtt{C2}$ are names for 
      complexity classes. Statements of this form are understood to indicate that 
      the two classes are equal with respect to every oracle. Complexity Zoology 
      uses the transitivity of equality to learn which classes are equal and then 
      chooses an official name for each class according to preferences specified in 
      the input file.
    \item \textbf{Expand operators:} Zoology understands each operator as a partial 
      function on the set of unique classes in the data set. Using the rules 
      specified in the input file for operators, the program expands each partial 
      function to be as large as possible.
    \item \textbf{Deduce:} The system applies its list of inference rules to deduce 
      inclusions, oracle separations, and open problems.
    \item \textbf{Postprocess:} The program prepares the expanded knowledge database
      for output. In particular, Complexity Zoology computes the relations that must
      be shown on the final diagram.
    \item \textbf{Output:} Zoology produces an HTML file with clickable diagrams 
      showing complexity class relationships in each modal world.
\end{enumerate}