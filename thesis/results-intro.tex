What follows is a survey of complexity theory that has been aided by the
Complexity Zoology software. It is similar in structure to Scott Aaronson's
Complexity Zoo, although with a much smaller collection of complexity
classes. The role of the software in producing this software has been threefold:
it has identified unanswered questions most useful to creating a complete
picture of the field, it has highlighted redundant information that is the
logical consequence of other data, and it has provided a high-level view of the
current state of knowledge.

Where possible, redundant data has been removed from the survey, although it is 
occasionally left in when it represents a particularly notable or foundational result. 
Results are explained in varying levels of detail with the intent of both providing a
roadmap of complexity theory and illustrating some key arguments and frequently used
techniques. It is hoped that this overview demonstrates the usefulness of a
computer-assisted approach to compiling knowledge about the relationships
between complexity classes.