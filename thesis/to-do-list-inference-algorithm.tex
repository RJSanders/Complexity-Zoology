\section{To-do List Inference Algorithm} \label{todo}

Both the primary inference engine and the propagation of operator partial
functions follow the same basic procedure. We begin with some database $D_0$,
which can be taken to be a set of propositions that are regarded as true. We
also assume that there is a set $R$ of inference rules, which are tuples of the
form
\[
\varphi_1,\varphi_2,\ldots,\varphi_n\vdash\psi
\]
for a positive integer $n$ (although for our purposes $n\leq 2$). The inference 
algorithm that Complexity Zoology employs is as follows:
\begin{enumerate}
\item Populate a list $L$ with the propositions in $D_0$, and set
  $D=\emptyset$.
\item While $L$ is nonempty, carry out the steps (3) through (6).
\item Remove the top proposition $\varphi$ from $L$.
\item If $\varphi\in D$, return to step (3).
\item Add $\varphi$ to the set $D$.
\item For each inference rule $\varphi_1,\varphi_2,\ldots,\varphi_n\vdash\psi$
  and all $\varphi_1^\prime,\varphi_2^\prime,\ldots,\varphi_{n-1}^\prime\in D$,
  check whether some permutation of
  $\varphi,\varphi_1^\prime,\ldots,\varphi_{n-1}^\prime$ matches
  $\varphi_1,\varphi_2,\ldots,\varphi_n$. If it does, append $\psi$ to $L$.
\end{enumerate}
The resulting database $D$ has $D_0$ as a subset and is closed under inference
rules. Moreover, this algorithm eventually terminates, because when a proposition
has been removed from $L$ once, it cannot again result in any additional
proposition being appended to $L$. Thus, the algorithm deduces all logical
consequences of the initial database $D_0$, and it does so faster than the naive
approach of repeatedly applying all inference rules to all the propositions in
$D$ until $D$ grows no further.
